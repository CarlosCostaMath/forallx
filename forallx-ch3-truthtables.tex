%!TEX root = forallx.tex
\chapter{Tabelas-verdade}
\label{ch.TruthTables}

Este capítulo apresenta uma maneira de avaliar sentenças e argumentos de SL. Embora possa ser trabalhoso, o método das tabelas-verdade é um procedimento puramente mecânico que não exige intuição nem qualquer insight especial.

\section{Conectivos verofuncionais}

Qualquer sentença não atômica de SL é composta de sentenças atômicas com conectivos sentenciais. O valor de verdade da sentença composta depende apenas do valor de verdade das sentenças atômicas que a compõem. Para saber o valor de verdade de $(D\eiff E)$, por exemplo, basta saber o valor de verdade de $D$ e o valor de verdade de $E$. Conectivos que funcionam dessa maneira são chamados de \define{verofuncionais}.

Neste capítulo, faremos uso do fato de que todos os operadores lógicos em SL são verofuncionais — isso torna possível construir tabelas-verdade para determinar as propriedades lógicas das sentenças. Você deve notar, no entanto, que isso não é possível para todas as linguagens. Em inglês, é possível formar uma nova sentença a partir de qualquer sentença mais simples \script{X} dizendo “É possível que \script{X}”. O valor de verdade dessa nova sentença não depende diretamente do valor de verdade de \script{X}. Mesmo que \script{X} seja falsa, talvez, em algum sentido, \script{X} \emph{pudesse} ter sido verdadeira — então a nova sentença seria verdadeira. Algumas linguagens formais, chamadas de \emph{lógicas modais}, possuem um operador de {possibilidade}. Em uma lógica modal, poderíamos traduzir “É possível que \script{X}” como {\large $\diamond$}\script{X}. No entanto, a capacidade de traduzir sentenças desse tipo tem um custo: o operador {\large $\diamond$} não é verofuncional e, portanto, lógicas modais não são tratáveis por tabelas-verdade.


\section{Tabelas-verdade completas}

O valor de verdade de sentenças que contêm apenas um conectivo é dado pela tabela-verdade característica daquele conectivo. No capítulo anterior, escrevemos as tabelas-verdade características com ‘T’ para verdadeiro e ‘F’ para falso. É importante notar, entretanto, que aqui não se trata de verdade em algum sentido profundo ou cósmico. Poetas e filósofos podem discutir longamente sobre a natureza e a importância da \emph{verdade}, mas as funções de verdade em SL são apenas regras que transformam valores de entrada em valores de saída. Para enfatizar isso, neste capítulo escreveremos ‘1’ e ‘0’ em vez de ‘T’ e ‘F’. Mesmo que interpretemos ‘1’ como significando ‘verdadeiro’ e ‘0’ como significando ‘falso’, computadores podem ser programados para preencher tabelas-verdade de modo puramente mecânico. Em uma máquina, ‘1’ pode significar que um registrador está ligado e ‘0’ que o registrador está desligado. Matematicamente, eles são apenas os dois valores possíveis que uma sentença de SL pode ter.

Aqui estão as tabelas-verdade dos conectivos de SL, escritas em termos de 1s e 0s.

\begin{table}[h!]
\begin{center}
\begin{tabular}{c|c}
\script{A} & \enot\script{A}\\
\hline
1 & 0\\
0 & 1 
\end{tabular}
\ \ \ \ 
\begin{tabular}{c|c|c|c|c|c}
\script{A} & \script{B} & \script{A}\eand\script{B} & \script{A}\eor\script{B} & \script{A}\eif\script{B} & \script{A}\eiff\script{B}\\
\hline
1 & 1 & 1 & 1 & 1 & 1\\
1 & 0 & 0 & 1 & 0 & 0\\
0 & 1 & 0 & 1 & 1 & 0\\
0 & 0 & 0 & 0 & 1 & 1
\end{tabular}
\end{center}
\caption{As tabelas-verdade características dos conectivos de SL.}
\label{table.CharacteristicTTs}
\end{table}

A tabela-verdade característica da conjunção, por exemplo, dá as condições de verdade para qualquer sentença da forma $(\script{A}\eand\script{B})$. Mesmo que os conjuntos \script{A} e \script{B} sejam sentenças longas e complicadas, a conjunção é verdadeira se, e somente se, tanto \script{A} quanto \script{B} forem verdadeiras. Considere a sentença $(H\eand I)\eif H$. Consideramos todas as combinações possíveis de verdadeiro e falso para $H$ e $I$, o que nos dá quatro linhas. Em seguida, copiamos os valores de verdade das letras sentenciais e os colocamos sob as letras na sentença.
\begin{center}
\begin{tabular}{c|c|@{\TTon}*{5}{c}@{\TToff}}
$H$&$I$&$(H$&\eand&$I)$&\eif&$H$\\
\hline
 1 & 1 & \TTbf{1} & & \TTbf{1} & & \TTbf{1}\\
 1 & 0 & \TTbf{1} & & \TTbf{0} & & \TTbf{1}\\
 0 & 1 & \TTbf{0} & & \TTbf{1} & & \TTbf{0}\\
 0 & 0 & \TTbf{0} & & \TTbf{0} & & \TTbf{0}
\end{tabular}
\end{center}
Agora considere a subsentença $H\eand I$. Trata-se de uma conjunção \script{A}\eand\script{B} com $H$ como \script{A} e $I$ como \script{B}. $H$ e $I$ são ambas verdadeiras na primeira linha. Como uma conjunção é verdadeira quando ambos os conjunções são verdadeiros, escrevemos 1 sob o símbolo de conjunção. Continuamos para as outras três linhas e obtemos:
\begin{center}
\begin{tabular}{c|c|@{\TTon}*{5}{c}@{\TToff}}
$H$&$I$&$(H$&\eand&$I)$&\eif&$H$\\
\hline
 & & \script{A} & \eand & \script{B} & & \\
 1 & 1 & 1 & \TTbf{1} & 1 & & 1\\
 1 & 0 & 1 & \TTbf{0} & 0 & & 1\\
 0 & 1 & 0 & \TTbf{0} & 1 & & 0\\
 0 & 0 & 0 & \TTbf{0} & 0 & & 0
\end{tabular}
\end{center}
A sentença inteira é um condicional \script{A}\eif\script{B}, com $(H\eand I)$ como \script{A} e $H$ como \script{B}. Na segunda linha, por exemplo, $(H\eand I)$ é falsa e $H$ é verdadeira. Como um condicional é verdadeiro quando o antecedente é falso, escrevemos 1 na segunda linha sob o símbolo do condicional. Continuamos nas outras três linhas e obtemos:
\begin{center}
\begin{tabular}{c|c|@{\TTon}*{5}{c}@{\TToff}}
$H$&$I$&$(H$&\eand&$I)$&\eif&$H$\\
\hline
 & &  & \script{A} &  &\eif &\script{B} \\
 1 & 1 &  & {1} &  &\TTbf{1} & 1\\
 1 & 0 &  & {0} &  &\TTbf{1} & 1\\
 0 & 1 &  & {0} &  &\TTbf{1} & 0\\
 0 & 0 &  & {0} &  &\TTbf{1} & 0
\end{tabular}
\end{center}
A coluna de 1s sob o condicional nos diz que a sentença \mbox{$(H \eand I)\eif I$} é verdadeira independentemente dos valores de verdade de $H$ e $I$. Eles podem ser verdadeiros ou falsos em qualquer combinação, e a sentença composta continua verdadeira. É crucial que tenhamos considerado todas as combinações possíveis. Se tivéssemos apenas uma tabela-verdade com duas linhas, não poderíamos ter certeza de que a sentença não seria falsa para alguma outra combinação de valores de verdade.

Neste exemplo, não repetimos todas as entradas em cada tabela sucessiva. Porém, ao escrever tabelas-verdade no papel, não é prático apagar colunas inteiras ou reescrever a tabela toda a cada passo. Embora fique mais apertado, a tabela-verdade pode ser escrita assim:
\begin{center}
\begin{tabular}{c|c|@{\TTon}*{5}{c}@{\TToff}}
$H$&$I$&$(H$&\eand&$I)$&\eif&$H$\\
\hline
 1 & 1 & 1 & {1} & 1 & 1 & 1\\
 1 & 0 & 1 & {0} & 0 & 1 & 1\\
 0 & 1 & 0 & {0} & 1 & 1 & 0\\
 0 & 0 & 0 & {0} & 0 & 1 & 0
\end{tabular}
\end{center}
A maior parte das colunas sob a sentença está lá apenas para organização. Quando você ficar mais ágil com tabelas-verdade, provavelmente não precisará mais copiar as colunas para cada letra sentencial. Em qualquer caso, o valor de verdade da sentença em cada linha é dado pela coluna sob o principal operador lógico da sentença; neste caso, a coluna sob o condicional.

Uma \define{tabela-verdade completa} tem uma linha para todas as combinações possíveis de 1 e 0 para todas as letras sentenciais. O tamanho da tabela-verdade completa depende do número de letras sentenciais diferentes na tabela. Uma sentença que contém apenas uma letra sentencial requer apenas duas linhas, como na tabela-verdade característica da negação. Isso é verdade mesmo se a mesma letra for repetida muitas vezes, como na sentença
$[(C\eiff C) \eif C] \eand \enot(C \eif C)$.
A tabela-verdade completa requer apenas duas linhas porque há apenas duas possibilidades: $C$ pode ser verdadeira ou pode ser falsa. Uma única letra sentencial nunca pode ser marcada como 1 e 0 na mesma linha. A tabela-verdade dessa sentença é:
\begin{center}
\begin{tabular}{c|@{\TTon}*{15}{c}@{\TToff}}
$C$&$[($&$C$&\eiff&$C$&$)$&\eif&$C$&$]$&\eand&\enot&$($&$C$&\eif&$C$&$)$\\
\hline
 1 &    & 1 &  1  & 1 &   & 1  & 1 & &\TTbf{0}&  0& &   1 &  1  & 1 &   \\
 0 &    & 0 &  1  & 0 &   & 0  & 0 & &\TTbf{0}&  0& &   0 &  1  & 0 &   \\
\end{tabular}
\end{center}
Olhando para a coluna sob o conectivo principal, vemos que a sentença é falsa nas duas linhas da tabela; isto é, ela é falsa independentemente de $C$ ser verdadeira ou falsa.

Uma sentença que contém duas letras sentenciais requer quatro linhas para uma tabela-verdade completa, como nas tabelas-verdade características e na tabela para $(H \eand I)\eif I$.

Uma sentença que contém três letras sentenciais requer oito linhas. Por exemplo:
\begin{center}
\begin{tabular}{c|c|c|@{\TTon}*{5}{c}@{\TToff}}
$M$&$N$&$P$&$M$&\eand&$(N$&\eor&$P)$\\
\hline
%           M        &     N   v   P
1 & 1 & 1 & 1 & \TTbf{1} & 1 & 1 & 1\\
1 & 1 & 0 & 1 & \TTbf{1} & 1 & 1 & 0\\
1 & 0 & 1 & 1 & \TTbf{1} & 0 & 1 & 1\\
1 & 0 & 0 & 1 & \TTbf{0} & 0 & 0 & 0\\
0 & 1 & 1 & 0 & \TTbf{0} & 1 & 1 & 1\\
0 & 1 & 0 & 0 & \TTbf{0} & 1 & 1 & 0\\
0 & 0 & 1 & 0 & \TTbf{0} & 0 & 1 & 1\\
0 & 0 & 0 & 0 & \TTbf{0} & 0 & 0 & 0
\end{tabular}
\end{center}
Dessa tabela, sabemos que a sentença $M\eand(N\eor P)$ pode ser verdadeira ou falsa, dependendo dos valores de verdade de $M$, $N$ e $P$.

Uma tabela-verdade completa para uma sentença que contém quatro letras sentenciais diferentes requer 16 linhas. Cinco letras, 32 linhas. Seis letras, 64 linhas. E assim por diante. De forma geral: se uma tabela-verdade completa tem $n$ letras sentenciais diferentes, então ela deve ter $2^n$ linhas.

Para preencher as colunas de uma tabela-verdade completa, comece pela letra sentencial mais à direita e alterne 1s e 0s. Na próxima coluna à esquerda, escreva dois 1s, depois dois 0s, e repita. Para a terceira letra sentencial, escreva quatro 1s seguidos de quatro 0s. Isso produz uma tabela-verdade com oito linhas, como a acima. Para uma tabela com 16 linhas, a próxima coluna de letras sentenciais deve ter oito 1s seguidos de oito 0s. Para uma tabela de 32 linhas, a coluna seguinte terá 16 1s seguidos de 16 0s. E assim por diante.


\section{Usando tabelas-verdade}

\subsection{Tautologias, contradições e sentenças contingentes}

Lembre que uma sentença em inglês é uma tautologia se ela \emph{tem de} ser verdadeira por questão de lógica. Com uma tabela-verdade completa, consideramos todas as maneiras pelas quais o mundo poderia ser. Se a sentença é verdadeira em todas as linhas da tabela-verdade completa, então ela é verdadeira por questão de lógica, independentemente de como o mundo é de fato.

Assim, uma sentença é uma \define{tautologia em SL} se a coluna sob o seu conectivo principal é 1 em todas as linhas de uma tabela-verdade completa.

Por outro lado, uma sentença é uma \define{contradição em SL} se a coluna sob o seu conectivo principal é 0 em todas as linhas de uma tabela-verdade completa.

Uma sentença é \define{contingente em SL} se não é nem tautologia nem contradição; isto é, se vale 1 em pelo menos uma linha e 0 em pelo menos outra linha.

Pelas tabelas da seção anterior, sabemos que $(H\eand I)\eif H$ é uma tautologia, que $[(C\eiff C) \eif C] \eand \enot(C \eif C)$ é uma contradição, e que $M \eand (N \eor P)$ é contingente.


\subsection{Equivalência lógica}

Duas sentenças são logicamente equivalentes em inglês se possuem o mesmo valor de verdade por questão de lógica. Mais uma vez, as tabelas-verdade nos permitem definir um conceito análogo para SL: duas sentenças são \define{logicamente equivalentes em SL} se possuem o mesmo valor de verdade em todas as linhas de uma tabela-verdade completa.

Considere as sentenças $\enot(A \eor B)$ e $\enot A \eand \enot B$. Elas são logicamente equivalentes? Para descobrir, construímos uma tabela-verdade.
\begin{center}
\begin{tabular}{c|c|@{\TTon}*{4}{c}@{\TToff}|@{\TTon}*{5}{c}@{\TToff}}
$A$&$B$&\enot&$(A$&\eor&$B)$&\enot&$A$&\eand&\enot&$B$\\
\hline
 1 & 1 & \TTbf{0} & 1 & 1 & 1 & 0 & 1 & \TTbf{0} & 0 & 1\\
 1 & 0 & \TTbf{0} & 1 & 1 & 0 & 0 & 1 & \TTbf{0} & 1 & 0\\
 0 & 1 & \TTbf{0} & 0 & 1 & 1 & 1 & 0 & \TTbf{0} & 0 & 1\\
 0 & 0 & \TTbf{1} & 0 & 0 & 0 & 1 & 0 & \TTbf{1} & 1 & 0
\end{tabular}
\end{center}
Observe as colunas dos conectivos principais; negação para a primeira sentença, conjunção para a segunda. Nas três primeiras linhas, ambas valem 0. Na última linha, ambas valem 1. Como coincidem em todas as linhas, as duas sentenças são logicamente equivalentes.


\subsection{Consistência}

Um conjunto de sentenças em inglês é consistente se é logicamente possível que sejam todas verdadeiras ao mesmo tempo.  
Um conjunto de sentenças é \define{logicamente consistente em SL} se existe pelo menos uma linha de uma tabela-verdade completa em que todas as sentenças são verdadeiras. Ele é \define{inconsistente} caso contrário.


\subsection{Validade}

Um argumento em inglês é válido se é logicamente impossível que as premissas sejam verdadeiras e a conclusão falsa ao mesmo tempo.  
Um argumento é \define{válido em SL} se não há linha de uma tabela-verdade completa em que todas as premissas sejam 1 e a conclusão seja 0; um argumento é \define{inválido em SL} se existe tal linha.

Considere este argumento:
\begin{earg}
\item[] $\enot L \eif (J \eor L)$
\item[] $\enot L$
\item[\therefore] $J$
\end{earg}
Ele é válido? Para descobrir, construímos uma tabela-verdade.
\begin{center}
\begin{tabular}{c|c|@{\TTon}*{6}{c}@{\TToff}|@{\TTon}*{2}{c}@{\TToff}|@{\TTon}c@{\TToff}}
$J$&$L$&\enot&$L$&\eif&$(J$&\eor&$L)$&\enot&L&J\\
\hline
%J   L   -   L      ->     (J   v   L)
 1 & 1 & 0 & 1 & \TTbf{1} & 1 & 1 & 1 & \TTbf{0} & 1 & \TTbf{1}\\
 1 & 0 & 1 & 0 & \TTbf{1} & 1 & 1 & 0 & \TTbf{1} & 0 & \TTbf{1}\\
 0 & 1 & 0 & 1 & \TTbf{1} & 0 & 1 & 1 & \TTbf{0} & 1 & \TTbf{0}\\
 0 & 0 & 1 & 0 & \TTbf{0} & 0 & 0 & 0 & \TTbf{1} & 0 & \TTbf{0}
\end{tabular}
\end{center}
Sim, o argumento é válido.  
A única linha em que ambas as premissas valem 1 é a segunda, e nessa linha a conclusão também vale 1.


\section{Tabelas-verdade parciais}

Para mostrar que uma sentença é uma tautologia, precisamos mostrar que ela vale 1 em todas as linhas. Portanto, precisamos de uma tabela-verdade completa. Para mostrar que uma sentença \emph{não} é uma tautologia, porém, basta uma linha: uma linha na qual a sentença valha 0. Assim, para mostrar que algo não é uma tautologia, é suficiente fornecer uma \emph{tabela-verdade parcial} de uma linha — independentemente de quantas letras sentenciais a sentença contenha.

Considere, por exemplo, a sentença $(U \eand T) \eif (S \eand W)$. Queremos mostrar que ela \emph{não} é uma tautologia fornecendo uma tabela-verdade parcial. Preenchemos 0 para a sentença inteira. O conectivo principal da sentença é um condicional. Para que o condicional seja falso, o antecedente deve ser verdadeiro (1) e o consequente falso (0). Preenchemos isso na tabela:
\begin{center}
\begin{tabular}{c|c|c|c|@{\TTon}*{7}{c}@{\TToff}}
$S$&$T$&$U$&$W$&$(U$&\eand&$T)$&\eif    &$(S$&\eand&$W)$\\
\hline
   &   &   &   &    &  1  &    &\TTbf{0}&    &   0 &   
\end{tabular}
\end{center}
Para que $(U\eand T)$ seja verdadeira, tanto $U$ quanto $T$ devem ser verdadeiras.
\begin{center}
\begin{tabular}{c|c|c|c|@{\TTon}*{7}{c}@{\TToff}}
$S$&$T$&$U$&$W$&$(U$&\eand&$T)$&\eif    &$(S$&\eand&$W)$\\
\hline
   & 1 & 1 &   &  1 &  1  & 1  &\TTbf{0}&    &   0 &   
\end{tabular}
\end{center}
Agora só precisamos fazer $(S\eand W)$ falsa. Para isso, precisamos tornar pelo menos uma entre $S$ e $W$ falsa. Podemos tornar ambas falsas, se quisermos. Tudo o que importa é que a sentença inteira acabe falsa nessa linha. Tomando uma decisão arbitrária, terminamos a tabela assim:
\begin{center}
\begin{tabular}{c|c|c|c|@{\TTon}*{7}{c}@{\TToff}}
$S$&$T$&$U$&$W$&$(U$&\eand&$T)$&\eif    &$(S$&\eand&$W)$\\
\hline
 0 & 1 & 1 & 0 &  1 &  1  & 1  &\TTbf{0}&  0 &   0 & 0  
\end{tabular}
\end{center}

Mostrar que algo é uma contradição exige uma tabela-verdade completa. Mostrar que algo \emph{não} é uma contradição exige apenas uma tabela-verdade parcial de uma linha, onde a sentença seja verdadeira nessa linha.

Uma sentença é contingente se não é nem tautologia nem contradição. Portanto, mostrar que uma sentença é contingente exige uma tabela-verdade parcial de \emph{duas linhas}: a sentença deve ser verdadeira em uma linha e falsa em outra. Por exemplo, podemos mostrar que a sentença acima é contingente com esta tabela:
\begin{center}
\begin{tabular}{c|c|c|c|@{\TTon}*{7}{c}@{\TToff}}
$S$&$T$&$U$&$W$&$(U$&\eand&$T)$&\eif    &$(S$&\eand&$W)$\\
\hline
 0 & 1 & 1 & 0 &  1 &  1  & 1  &\TTbf{0}&  0 &   0 & 0 \\
 0 & 1 & 0 & 0 &  0 &  0  & 1  &\TTbf{1}&  0 &   0 & 0
\end{tabular}
\end{center}
Observe que há muitas combinações de valores de verdade que tornariam a sentença verdadeira, então há muitas maneiras de escrever a segunda linha.

Mostrar que uma sentença \emph{não} é contingente exige fornecer uma tabela-verdade completa, porque isso requer mostrar que a sentença é uma tautologia ou uma contradição. Se você não sabe se uma dada sentença é contingente, então não sabe se vai precisar de uma tabela completa ou parcial. Você pode sempre começar construindo uma tabela-verdade completa. Se, ao completar algumas linhas, você mostrar que a sentença é contingente, pode parar. Caso contrário, termine a tabela. Embora duas linhas cuidadosamente escolhidas sejam suficientes para mostrar que uma sentença contingente é contingente, não há problema em preencher mais linhas.

Mostrar que duas sentenças são logicamente equivalentes exige fornecer uma tabela-verdade completa. Mostrar que duas sentenças \emph{não} são logicamente equivalentes exige apenas uma tabela-verdade parcial de uma linha: construa uma linha em que uma sentença seja verdadeira e a outra falsa.

Mostrar que um conjunto de sentenças é consistente exige fornecer uma linha de uma tabela-verdade na qual todas as sentenças sejam verdadeiras. O restante da tabela é irrelevante, portanto uma tabela-verdade parcial de uma linha basta. Mostrar que um conjunto de sentenças é inconsistente, por outro lado, exige uma tabela-verdade completa: é preciso mostrar que, em todas as linhas, pelo menos uma das sentenças é falsa.

Mostrar que um argumento é válido exige uma tabela-verdade completa. Mostrar que um argumento é \emph{inválido} exige apenas uma tabela-verdade de uma linha: se você puder produzir uma linha em que as premissas sejam todas verdadeiras e a conclusão falsa, o argumento é inválido.

Segue uma tabela que resume quando é necessária uma tabela-verdade completa e quando uma tabela-verdade parcial é suficiente.

\begin{table}[h!]
\begin{center}
\begin{tabular}{c|c|c|}
\cline{2-3}
 & SIM & NÃO\\
\cline{2-3}
tautologia? & tabela-verdade completa & tabela-verdade parcial de uma linha\\
contradição? &  tabela-verdade completa  & tabela-verdade parcial de uma linha\\
contingente? & tabela-verdade parcial de duas linhas & tabela-verdade completa\\
equivalente? & tabela-verdade completa & tabela-verdade parcial de uma linha\\
consistente? & tabela-verdade parcial de uma linha & tabela-verdade completa\\
válido? & tabela-verdade completa & tabela-verdade parcial de uma linha\\
\cline{2-3}
\end{tabular}
\end{center}
\caption{Você precisa de uma tabela-verdade completa ou parcial? Depende do que está tentando mostrar.}
\label{table.CompleteVsPartial}
\end{table}


\practiceproblems
Se quiser prática adicional, você pode construir tabelas-verdade para qualquer uma das sentenças e argumentos dos exercícios do capítulo anterior.


\solutions
\problempart
\label{pr.TT.TTorC}
Determine se cada sentença é uma tautologia, uma contradição ou uma sentença contingente. Justifique sua resposta com uma tabela-verdade completa ou parcial, conforme apropriado.
\begin{earg}
\item $A \eif A$ %tautologia
\item $\enot B \eand B$ %contradição
\item $C \eif\enot C$ %contingente
\item $\enot D \eor D$ %tautologia
\item $(A \eiff B) \eiff \enot(A\eiff \enot B)$ %tautologia
\item $(A\eand B) \eor (B\eand A)$ %contingente
\item $(A \eif B) \eor (B \eif A)$ %tautologia
\item $\enot[A \eif (B \eif A)]$ %contradição
\item $(A \eand B) \eif (B \eor A)$  %tautologia
\item $A \eiff [A \eif (B \eand \enot B)]$ %contradição
\item $\enot(A \eor B) \eiff (\enot A \eand \enot B)$ %tautologia
\item $\enot(A\eand B) \eiff A$ %contingente
\item $\bigl[(A\eand B) \eand\enot(A\eand B)\bigr] \eand C$ %contradição
\item $A\eif(B\eor C)$ %contingente
\item $[(A \eand B) \eand C] \eif B$ %tautologia
\item $(A \eand\enot A) \eif (B \eor C)$ %tautologia
\item $\enot\bigl[(C\eor A) \eor B\bigr]$ %contingente
\item $(B\eand D) \eiff [A \eiff(A \eor C)]$%contingente
\end{earg}


\solutions
\problempart
\label{pr.TT.equiv}
Determine se cada par de sentenças é logicamente equivalente. Justifique sua resposta com uma tabela-verdade completa ou parcial, conforme apropriado.
\begin{earg}
\item $A$, $\enot A$ %Não
\item $A$, $A \eor A$ %Sim
\item $A\eif A$, $A \eiff A$ %Não
\item $A \eor \enot B$, $A\eif B$ %Não
\item $A \eand \enot A$, $\enot B \eiff B$ %Sim
\item $\enot(A \eand B)$, $\enot A \eor \enot B$ %Sim
\item $\enot(A \eif B)$, $\enot A \eif \enot B$ %Não
\item $(A \eif B)$, $(\enot B \eif \enot A)$ %Sim
\item $[(A \eor B) \eor C]$, $[A \eor (B \eor C)]$ %Sim
\item $[(A \eor B) \eand C]$, $[A \eor (B \eand C)]$ %Não
\end{earg}

\solutions
\problempart
\label{pr.TT.consistent}
Determine se cada conjunto de sentenças é consistente ou inconsistente. Justifique sua resposta com uma tabela-verdade completa ou parcial, conforme apropriado.
\begin{earg}
\item $A\eif A$, $\enot A \eif \enot A$, $A\eand A$, $A\eor A$ %consistente
\item $A \eand B$, $C\eif \enot B$, $C$ %inconsistente
\item $A\eor B$, $A\eif C$, $B\eif C$ %consistente
\item $A\eif B$, $B\eif C$, $A$, $\enot C$ %inconsistente
\item $B\eand(C\eor A)$, $A\eif B$, $\enot(B\eor C)$  %inconsistente
\item $A \eor B$, $B\eor C$, $C\eif \enot A$ %consistente
\item $A\eiff(B\eor C)$, $C\eif \enot A$, $A\eif \enot B$ %consistente
\item $A$, $B$, $C$, $\enot D$, $\enot E$, $F$ %consistente
\end{earg}

\solutions
\problempart
\label{pr.TT.valid}
Determine se cada argumento é válido ou inválido. Justifique sua resposta com uma tabela-verdade completa ou parcial, conforme apropriado.
\begin{earg}
\item $A\eif A$, \therefore\ $A$ %inválido
\item $A\eor\bigl[A\eif(A\eiff A)\bigr]$, \therefore\ A %inválido
\item $A\eif(A\eand\enot A)$, \therefore\ $\enot A$ %válido
\item $A\eiff\enot(B\eiff A)$, \therefore\ $A$ %inválido
\item $A\eor(B\eif A)$, \therefore\ $\enot A \eif \enot B$ %válido
\item $A\eif B$, $B$, \therefore\ $A$ %inválido
\item $A\eor B$, $B\eor C$, $\enot A$, \therefore\ $B \eand C$ %inválido
\item $A\eor B$, $B\eor C$, $\enot B$, \therefore\ $A \eand C$ %válido
\item $(B\eand A)\eif C$, $(C\eand A)\eif B$, \therefore\ $(C\eand B)\eif A$ %inválido
\item $A\eiff B$, $B\eiff C$, \therefore\ $A\eiff C$ %válido
\end{earg}

\solutions
\problempart
\label{pr.TT.concepts}
Responda a cada uma das questões abaixo e justifique sua resposta.
\begin{earg}
\item Suponha que \script{A} e \script{B} sejam logicamente equivalentes. O que você pode dizer sobre $\script{A}\eiff\script{B}$?
%\script{A} e \script{B} têm o mesmo valor de verdade em todas as linhas de uma tabela-verdade completa, portanto $\script{A}\eiff\script{B}$ é verdadeira em todas as linhas. É uma tautologia.
\item Suponha que $(\script{A}\eand\script{B})\eif\script{C}$ seja contingente. O que você pode dizer sobre o argumento “\script{A}, \script{B}, \therefore\script{C}”?
%A sentença é falsa em alguma linha de uma tabela-verdade completa. Nessa linha, \script{A} e \script{B} são verdadeiras e \script{C} é falsa. Logo, o argumento é inválido.
\item Suponha que $\{\script{A},\script{B}, \script{C}\}$ seja inconsistente. O que você pode dizer sobre $(\script{A}\eand\script{B}\eand\script{C})$?
%Como não há linha de uma tabela-verdade completa em que as três sentenças sejam verdadeiras, a conjunção é falsa em todas as linhas. Logo, é uma contradição.
\item Suponha que \script{A} seja uma contradição. O que você pode dizer sobre o argumento “\script{A}, \script{B}, \therefore\script{C}”?
%Como \script{A} é falsa em todas as linhas de uma tabela-verdade completa, não há linha em que \script{A} e \script{B} sejam verdadeiras e \script{C} seja falsa. Logo, o argumento é válido.
\item Suponha que \script{C} seja uma tautologia. O que você pode dizer sobre o argumento “\script{A}, \script{B}, \therefore\script{C}”?
%Como \script{C} é verdadeira em todas as linhas de uma tabela-verdade completa, não há linha em que \script{A} e \script{B} sejam verdadeiras e \script{C} seja falsa. Logo, o argumento é válido.
\item Suponha que \script{A} e \script{B} sejam logicamente equivalentes. O que você pode dizer sobre $(\script{A}\eor\script{B})$?
%Não muito. $(\script{A}\eor\script{B})$ é uma tautologia se \script{A} e \script{B} forem tautologias; é uma contradição se forem contradições; é contingente se forem contingentes.
\item Suponha que \script{A} e \script{B} \emph{não} sejam logicamente equivalentes. O que você pode dizer sobre $(\script{A}\eor\script{B})$?
%\script{A} e \script{B} têm valores de verdade diferentes em pelo menos uma linha de uma tabela-verdade completa, e $(\script{A}\eor\script{B})$ será verdadeira nessa linha. Em outras linhas, pode ser verdadeira ou falsa. Logo, $(\script{A}\eor\script{B})$ é ou uma tautologia ou contingente; ela \emph{não} é uma contradição.
\end{earg}

\problempart
\label{pr.altConnectives}
Poderíamos dispensar o bicondicional (\eiff) da linguagem. Se fizéssemos isso, ainda poderíamos escrever `$A\eiff B$’ para tornar as sentenças mais legíveis, mas isso seria apenas uma abreviação para $(A\eif B) \eand (B\eif A)$. A linguagem resultante seria formalmente equivalente a SL, já que $A\eiff B$ e $(A\eif B) \eand (B\eif A)$ são logicamente equivalentes em SL. Se valorizássemos a simplicidade formal acima da riqueza expressiva, poderíamos substituir mais conectivos por convenções notacionais e ainda ter uma linguagem equivalente a SL.

Existem várias linguagens equivalentes com apenas dois conectivos. Seria suficiente ter apenas negação e o condicional material. Mostre isso escrevendo sentenças logicamente equivalentes a cada uma das seguintes usando apenas parênteses, letras sentenciais, negação (\enot) e o condicional material (\eif).
\begin{earg}
\item\leftsolutions\ $A\eor B$
%$\enot A \eif B$
\item\leftsolutions\ $A\eand B$
%$\enot(A \eif \enot B)$
\item\leftsolutions\ $A\eiff B$
%$\enot [(A\eif B) \eif \enot(B\eif A)]$
\end{earg}
%...
% Sair do ambiente {earg} para dar novas instruções. 

Poderíamos ter uma linguagem equivalente a SL com apenas negação e disjunção como conectivos. Mostre isso: usando apenas parênteses, letras sentenciais, negação (\enot) e disjunção (\eor), escreva sentenças logicamente equivalentes a cada uma das seguintes.
% Retomar o ambiente {earg} e restaurar o contador.
%...
\begin{earg}
\setcounter{eargnum}{\arabic{OLDeargnum}}
\item $A \eand B$
%$\enot(\enot A \eor \enot B)$
\item $A \eif B$
%$\enot A \eor B$
\item $A \eiff B$
%$\enot(\enot A \eor \enot B) \eor \enot(A \eor B)$
\end{earg}
%...
O \emph{traço de Sheffer} é um conectivo lógico com a seguinte tabela-verdade característica:
\begin{center}
\begin{tabular}{c|c|c}
\script{A} & \script{B} & \script{A}$|$\script{B}\\
\hline
1 & 1 & 0\\
1 & 0 & 1\\
0 & 1 & 1\\
0 & 0 & 1
\end{tabular}
\end{center}
%...
\begin{earg}
\setcounter{eargnum}{\arabic{OLDeargnum}}
\item Escreva uma sentença usando os conectivos de SL que seja logicamente equivalente a $(A|B)$.
\end{earg}
%...
Toda sentença escrita usando um conectivo de SL pode ser reescrita como uma sentença logicamente equivalente usando um ou mais traços de Sheffer. Usando apenas o traço de Sheffer, escreva sentenças equivalentes a cada uma das seguintes.
%...
\begin{earg}
\setcounter{eargnum}{\arabic{OLDeargnum}}
\item $\enot A$
\item $(A\eand B)$
\item $(A\eor B)$
\item $(A\eif B)$
\item $(A\eiff B)$
\end{earg}
