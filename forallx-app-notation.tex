%!TEX root = forallx.tex
\chapter[Outra notação simbólica]{Notação simbólica}
\label{app.notation}
\markright{apêndice: notação simbólica}

Na história da lógica formal, diferentes símbolos foram usados em diferentes épocas e por diferentes autores. Muitas vezes, os autores eram forçados a usar notação que seus impressores pudessem compor.

Em um sentido, os símbolos usados para várias constantes lógicas são arbitrários. Não há nada escrito no céu que diga que `\enot' deve ser o símbolo para negação verofuncional. Poderíamos ter especificado um símbolo diferente para desempenhar esse papel. Uma vez que demos definições para fórmulas bem formadas (fbf) e para verdade em nossas linguagens lógicas, no entanto, usar `\enot' não é mais arbitrário. Esse é o símbolo para negação neste livro, e portanto é o símbolo para negação ao escrever sentenças em nossas linguagens LV ou LQ.

Este apêndice apresenta alguns símbolos comuns, para que você possa reconhecê-los se os encontrar em um artigo ou em outro livro.
\marginpar{
\begin{tabular}{rl}
\multicolumn{2}{c}{resumo dos símbolos}\\
negação & $\neg$, ${\sim}$\\
conjunção & $\&$, $\wedge$,	{\scriptsize\textbullet}\\
disjunção & $\vee$\\
condicional & $\rightarrow$, $\supset$\\
bicondicional & $\leftrightarrow$, $\equiv$
\end{tabular}
}

\paragraph{Negação} Dois símbolos comumente usados são a \emph{enxada}, `$\neg$', e o \emph{til}, `${\sim}$.' Em alguns sistemas formais mais avançados é necessário distinguir entre dois tipos de negação; a distinção é às vezes representada usando ambos `$\neg$' e `${\sim}$.'

\paragraph{Disjunção} O símbolo `$\vee$' é tipicamente usado para simbolizar disjunção inclusiva. %Em alguns sistemas, a disjunção é escrita como adição.

\paragraph{Conjunção}
A conjunção é frequentemente simbolizada com o \emph{e comercial}, `{\&}.' O e comercial é na verdade uma forma decorativa da palavra latina `et' que significa `e'; é comumente usado na escrita em português. Como um símbolo em um sistema formal, o e comercial não é a palavra `e'; seu significado é dado pela semântica formal da linguagem. Talvez para evitar essa confusão, alguns sistemas usam um símbolo diferente para conjunção. Por exemplo, `$\wedge$' é uma contraparte do símbolo usado para disjunção. Às vezes um ponto único, `{\scriptsize\textbullet}', é usado. Em alguns textos mais antigos, não há símbolo para conjunção; `$A$ e $B$' é simplesmente escrito `$AB$.'

\paragraph{Condicional Material} Há dois símbolos comuns para o condicional material: a \emph{seta}, `$\rightarrow$', e o \emph{gancho}, `$\supset$.'

\paragraph{Bicondicional Material} A \emph{seta de duas pontas}, `$\leftrightarrow$', é usada em sistemas que usam a seta para representar o condicional material. Sistemas que usam o gancho para o condicional tipicamente usam a \emph{barra tripla}, `$\equiv$', para o bicondicional.

\paragraph{Quantificadores} O quantificador universal é tipicamente simbolizado como um A de cabeça para baixo, `$\forall$', e o quantificador existencial como um E ao contrário, `$\exists$.' Em alguns textos, não há um símbolo separado para o quantificador universal. Em vez disso, a variável é simplesmente escrita entre parênteses em frente à fórmula que ela liga. Por exemplo, `todo $x$ é $P$' é escrito $(x)Px$.

Em alguns sistemas, os quantificadores são simbolizados com versões maiores dos símbolos usados para conjunção e disjunção. Embora expressões quantificadas não possam ser traduzidas em expressões sem quantificadores, há uma conexão conceitual entre o quantificador universal e a conjunção e entre o quantificador existencial e a disjunção. Considere a sentença $\exists x Px$, por exemplo. Ela significa que \emph{ou} o primeiro membro do UD é um $P$, \emph{ou} o segundo é, \emph{ou} o terceiro é, {\ldots}. Tal sistema usa o símbolo `$\bigvee$' em vez de `$\exists$.'




\section*{Notação polonesa}

Esta seção discute brevemente a lógica sentencial na notação polonesa, um sistema de notação introduzido no final dos anos 1920 pelo lógico polonês Jan {\L}ukasiewicz.

Letras minúsculas são usadas como letras de sentença. A letra maiúscula $N$ é usada para negação. $A$ é usada para disjunção, $K$ para conjunção, $C$ para o condicional, $E$ para o bicondicional. (`A' é para alternação, outro nome para disjunção lógica. `E' é para equivalência.)
\marginpar{
\begin{tabular}{cc}
notação & notação polonesa\\
de LV & \\
\enot & $N$\\
\eand & $K$\\
\eor & $A$\\
\eif & $C$\\
\eiff & $E$
\end{tabular}
}

Na notação polonesa, um conectivo binário é escrito \emph{antes} das duas sentenças que ele conecta. Por exemplo, a sentença $A\eand B$ de LV seria escrita $Kab$ em notação polonesa.

As sentenças $\enot A\eif B$ e $\enot (A\eif B)$ são muito diferentes; o operador lógico principal da primeira é o condicional, mas o conectivo principal da segunda é a negação. Em LV, mostramos isso colocando parênteses em torno do condicional na segunda sentença. Na notação polonesa, parênteses nunca são necessários. O conectivo mais à esquerda é sempre o conectivo principal. A primeira sentença seria simplesmente escrita $CNab$ e a segunda $NCab$.

Essa característica da notação polonesa significa que é possível avaliar sentenças simplesmente trabalhando através dos símbolos da direita para a esquerda. Se você estivesse construindo uma tabela-verdade para $NKab$, por exemplo, você primeiro consideraria os valores-verdade atribuídos a $b$ e $a$, então consideraria sua conjunção, e então negaria o resultado. A regra geral para o que avaliar a seguir em LV não é nem de longe tão simples. Em LV, a tabela-verdade para $\enot(A\eand B)$ requer olhar para $A$ e $B$, então olhar no meio da sentença para a conjunção, e então no início da sentença para a negação. Como a ordem das operações pode ser especificada mais mecanicamente na notação polonesa, variantes da notação polonesa são usadas como a estrutura interna para muitas linguagens de programação de computador.
