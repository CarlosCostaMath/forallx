%!TEX root = forallx.tex
\chapter[Soluções para exercícios selecionados]{Soluções para exercícios selecionados}
\label{app.solutions}

Muitos dos exercícios podem ser respondidos corretamente de diferentes maneiras. Quando esse for o caso, a solução aqui representa uma possível resposta correta.

\solutionsection{ch.intro}{pr.MartianGiraffes}
\begin{earg}
%\nextSeq %G2, G3, and G4
\item consistente
%\noSeq %G1, G3, and G4
\item inconsistente
%\nextSeq %G1, G2, and G4
\item consistente
%\lastSeq %G1, G2, and G3
\item consistente
%são consistentes.
\end{earg}

\solutionsection{ch.intro}{pr.EnglishCombinations}
%\begin{earg}
%\item Um argumento válido que tem uma premissa falsa e uma premissa verdadeira
%possível
\nextSeq
%\item Um argumento válido que tem uma conclusão falsa
%possível
\nextSeq
%\item Um argumento válido, cuja conclusão é uma contradição
%possível
\nextSeq
%\item Um argumento inválido, cuja conclusão é uma tautologia
%não possível
\noSeq
%\item Uma tautologia que é contingente
%não possível
\noSeq
%\item Duas sentenças logicamente equivalentes, ambas tautologias
%possível
\nextSeq
%\item Duas sentenças logicamente equivalentes, uma das quais é uma tautologia e a outra é contingente
%não possível
\noSeq
%\item Duas sentenças logicamente equivalentes que juntas são um conjunto inconsistente
%possível
\nextSeq
%Considere duas contradições. Como questão de lógica, ambas são necessariamente falsas. Então elas têm o mesmo valor de verdade: falso. Isso as torna logicamente equivalentes.
%\item Um conjunto consistente de sentenças que contém uma contradição
%não possível
\noSeq
%\item Um conjunto inconsistente de sentenças que contém uma tautologia
%possível
\lastSeq
%são possíveis.

\solutionsection{ch.LS}{pr.monkeysuits}
\begin{earg}
\item $\enot M$
\item $M \eor \enot M$
\item $G \eor C$
\item $\enot C \eand \enot G$
\item $C \eif (\enot G \eand \enot M)$
\item $M \eor (C \eor G)$
\end{earg}


\solutionsection{ch.LS}{pr.avacareer}
\begin{earg}
\item $E_1 \eand E_2$
\item $F_1 \eif S_1$
\item $F_1 \eor E_1$
\item $E_2 \eand \enot S_2$
\item $\enot E_1 \eand \enot E_2$
\item $E_1 \eand E_2 \eand \enot(S_1 \eor S_2)$
\item $S_2 \eif F_2$
\item $(\enot E_1 \eif \enot E_2) \eand (E_1 \eif E_2)$
\item $S_1 \eiff \enot S_2$
\item $(E_2 \eand F_2) \eif S_2$
\item $\enot(E_2 \eand F_2)$
\item $(F_1 \eand F_2) \eiff (\enot E_1 \eand \enot E_2)$
\end{earg}

\solutionsection{ch.LS}{pr.spies}
\begin{ekey}
\item[A:] Alice é uma espiã.
\item[B:] Bob é uma espiã.
\item[C:] O código foi quebrado.
\item[G:] A embaixada alemã estará em polvorosa.
\end{ekey}
\begin{earg}
\item %Alice e Bob são ambos espiões.
$A \eand B$
\item %Se Alice ou Bob é uma espiã, então o código foi quebrado.
$(A \eor B) \eif C$
\item %Se nem Alice nem Bob são espiões, então o código permanece intacto.
$\enot(A \eor B) \eif \enot C$
\item %A embaixada alemã estará em polvorosa, a menos que alguém tenha quebrado o código.
$G \eor C$
\item %Ou o código foi quebrado ou não foi, mas a embaixada alemã estará em polvorosa de qualquer maneira.
$(C \eor \enot C) \eand G$
\item %Ou Alice ou Bob é uma espiã, mas não ambos.
$(A \eor B) \eand \enot(A \eand B)$
\end{earg}


\solutionsection{ch.LS}{pr.wiffSL}
\begin{earg}
\item %$(A)$
(a) não (b) não
\item %$J_{374} \eor \enot J_{374}$
(a) não (b) sim
\item %$\enot \enot \enot \enot F$
(a) sim (b) sim
\item %$\enot \eand S$
(a) não (b) não
\item %$(G \eand \enot G)$
(a) sim (b) sim
\item %$\script{A} \eif \script{A}$
(a) não (b) não
\item %$(A \eif (A \eand \enot F)) \eor (D \eiff E)$
(a) não (b) sim
\item %$[(Z \eiff S) \eif W] \eand [J \eor X]$
(a) não (b) sim
\item %$(F \eiff \enot D \eif J) \eor (C \eand D)$
(a) não (b) não
\end{earg}




\solutionsection{ch.TruthTables}{pr.TT.TTorC}
\begin{earg}
\item tautologia
\item contradição
\item contingente
\item tautologia
\item tautologia
\item contingente
\item tautologia
\item contradição
\item tautologia
\item contradição
\item tautologia
\item contingente
\item contradição
\item contingente
\item tautologia
\item tautologia
\item contingente
\item contingente
\end{earg}

\solutionsection{ch.TruthTables}{pr.TT.equiv}
%\begin{earg}
%\item não equivalente
\noSeq
%\item equivalente
\nextSeq
%\item equivalente
\nextSeq
%\item não equivalente
\noSeq
%\item equivalente
\nextSeq
%\item equivalente
\nextSeq
%\item não equivalente
\noSeq
%\item equivalente
\nextSeq
%\item equivalente
\lastSeq
%\item não equivalente
\noSeq
%são logicamente equivalentes.

\solutionsection{ch.TruthTables}{pr.TT.consistent}
%\item $A\eif A$, $\enot A \eif \enot A$, $A\eand A$, $A\eor A$ %consistente
\nextSeq
%\item $A \eand B$, $C\eif \enot B$, $C$ %inconsistente
\noSeq
%\item $A\eor B$, $A\eif C$, $B\eif C$ %consistente
\nextSeq
%\item $A\eif B$, $B\eif C$, $A$, $\enot C$ %inconsistente
\noSeq
%\item $B\eand(C\eor A)$, $A\eif B$, $\enot(B\eor C)$  %inconsistente
\noSeq
%\item $A \eor B$, $B\eor C$, $C\eif \enot A$ %consistente
\nextSeq
%\item $A\eiff(B\eor C)$, $C\eif \enot A$, $A\eif \enot B$ %consistente
\nextSeq
%\item $A$, $B$, $C$, $\enot D$, $\enot E$, $F$ %consistente
\lastSeq
são consistentes.

\solutionsection{ch.TruthTables}{pr.TT.valid}
%\item $A\eif A$, \therefore\ $A$ %inválido
\noSeq
%\item $A\eor\bigl[A\eif(A\eiff A)\bigr]$, \therefore\ A %inválido
\noSeq
%\item $A\eif(A\eand\enot A)$, \therefore\ $\enot A$ %válido
\nextSeq
%\item $A\eiff\enot(B\eiff A)$, \therefore\ $A$ %inválido
\noSeq
%\item $A\eor(B\eif A)$, \therefore\ $\enot A \eif \enot B$ %válido
\nextSeq
%\item $A\eif B$, $B$, \therefore\ $A$ %inválido
\noSeq
%\item $A\eor B$, $B\eor C$, $\enot A$, \therefore\ $B \eand C$ %inválido
\noSeq
%\item $A\eor B$, $B\eor C$, $\enot B$, \therefore\ $A \eand C$ %válido
\nextSeq
%\item $(B\eand A)\eif C$, $(C\eand A)\eif B$, \therefore\ $(C\eand B)\eif A$ %inválido
\noSeq
%\item $A\eiff B$, $B\eiff C$, \therefore\ $A\eiff C$ %válido
\lastSeq
são válidos.


\solutionsection{ch.TruthTables}{pr.TT.concepts}
\begin{earg}
\item %Suponha que \script{A} e \script{B} são logicamente equivalentes. O que você pode dizer sobre $\script{A}\eiff\script{B}$?
\script{A} e \script{B} têm o mesmo valor de verdade em cada linha de uma tabela-verdade completa, então $\script{A}\eiff\script{B}$ é verdadeira em cada linha. É uma tautologia.
\item %Suponha que $(\script{A}\eand\script{B})\eif\script{C}$ é contingente. O que você pode dizer sobre o argumento ``\script{A}, \script{B}, \therefore\script{C}''?
A sentença é falsa em alguma linha de uma tabela-verdade completa. Nessa linha, \script{A} e \script{B} são verdadeiras e \script{C} é falsa. Então o argumento é inválido.
\item %Suponha que $\{\script{A},\script{B}, \script{C}\}$ é inconsistente. O que você pode dizer sobre $(\script{A}\eand\script{B}\eand\script{C})$?
Como não há linha de uma tabela-verdade completa na qual todas as três sentenças são verdadeiras, a conjunção é falsa em cada linha. Então é uma contradição.
\item %Suponha que \script{A} é uma contradição. O que você pode dizer sobre o argumento ``\script{A}, \script{B}, \therefore\script{C}''?
Como \script{A} é falsa em cada linha de uma tabela-verdade completa, não há linha na qual \script{A} e \script{B} são verdadeiras e \script{C} é falsa. Então o argumento é válido.
\item %Suponha que \script{C} é uma tautologia. O que você pode dizer sobre o argumento ``\script{A}, \script{B}, \therefore\script{C}''?
Como \script{C} é verdadeira em cada linha de uma tabela-verdade completa, não há linha na qual \script{A} e \script{B} são verdadeiras e \script{C} é falsa. Então o argumento é válido.
\item %Suponha que \script{A} e \script{B} são logicamente equivalentes. O que você pode dizer sobre $(\script{A}\eor\script{B})$?
Não muito. $(\script{A}\eor\script{B})$ é uma tautologia se \script{A} e \script{B} são tautologias; é uma contradição se são contradições; é contingente se são contingentes.
\item %Suponha que \script{A} e \script{B} são \emph{não} logicamente equivalentes. O que você pode dizer sobre $(\script{A}\eor\script{B})$?
\script{A} e \script{B} têm valores de verdade diferentes em pelo menos uma linha de uma tabela-verdade completa, e $(\script{A}\eor\script{B})$ será verdadeira nessa linha. Em outras linhas, pode ser verdadeira ou falsa. Então $(\script{A}\eor\script{B})$ ou é uma tautologia ou é contingente; é \emph{não} uma contradição.
\end{earg}


\solutionsection{ch.TruthTables}{pr.altConnectives}
\begin{earg}
\item %$A\eor B$
$\enot A \eif B$
\item %$A\eand B$
$\enot(A \eif \enot B)$
\item %$A\eiff B$
$\enot [(A\eif B) \eif \enot(B\eif A)]$
%\item %$A \eand B$
%$\enot(\enot A \eor \enot B)$
%\item %$A \eif B$
%$\enot A \eor B$
%\item %$A \eiff B$
%$\enot(\enot A \eor \enot B) \eor \enot(A \eor B)$
\end{earg}



\solutionsection{ch.LQ}{pr.QLalligators}
\begin{earg}
\item %Amos, Bouncer e Cleo todos moram no zoológico.
$Za \eand Zb \eand Zc$
\item %Bouncer é um réptil, mas não um jacaré. 
$Rb \eand \enot Ab$
\item %Se Cleo ama Bouncer, então Bouncer é um macaco. 
$Lcb \eif Mb$
\item %Se ambos Bouncer e Cleo são jacarés, então Amos ama ambos.
$(Ab \eand Ac)\eif(Lab \eand Lac)$
\item %Algum réptil mora no zoológico. 
$\exists x(Rx \eand Zx)$
\item %Todo jacaré é um réptil. 
$\forall x(Ax \eif Rx)$
\item %Qualquer animal que mora no zoológico é ou um macaco ou um jacaré. 
$\forall x\bigl[Zx \eif (Mx \eor Ax)\bigr]$
\item %Há répteis que não são jacarés.
$\exists x(Rx \eand \enot Ax)$
\item %Cleo ama um réptil.
$\exists x(Rx \eand Lcx)$
\item %Bouncer ama todos os macacos que moram no zoológico.
$\forall x\bigl[(Mx \eand Zx) \eif Lbx\bigr]$
\item %Todos os macacos que Amos ama o amam de volta.
$\forall x\bigl[(Mx \eand Lax) \eif Lxa\bigr]$
\item %Se algum animal é um réptil, então Amos é.
$\exists x Rx \eif Ra$
\item %Se algum animal é um jacaré, então é um réptil.
$\forall x(Ax \eif Rx)$
\item %Todo macaco que Cleo ama também é amado por Amos.
$\forall x\bigl[(Mx \eand Lcx) \eif Lax\bigr]$
\item %Há um macaco que ama Bouncer, mas infelizmente Bouncer não reciprocra este amor.
$\exists x(Mx \eand Lxb \eand \enot Lbx)$
\end{earg}


\solutionsection{ch.LQ}{pr.QLcandies}
\begin{earg}
\item %Boris nunca experimentou nenhum doce.
$\enot\exists x Tx$
\item %O maçapão é sempre feito com açúcar.
$\forall x(Mx \eif Sx)$
\item %Algum doce é sem açúcar.
$\exists x \enot Sx$
\item %O melhor doce é o chocolate.
$\exists x[Cx \eand \enot\exists y Byx]$
\item %Nenhum doce é melhor que si mesmo.
$\enot \exists x Bxx$
\item %Boris nunca experimentou chocolate sem açúcar.
$\enot \exists x(Cx \eand \enot Sx \eand Tx)$
\item %Boris experimentou maçapão e chocolate, mas nunca juntos.
$\exists x(Cx \eand Tx) \eand \exists x(Mx \eand Tx) \eand \enot \exists x(Cx \eand Mx \eand Tx)$
%\item Boris não experimentou nada que seja melhor que maçapão sem açúcar.
%$\enot\exists x[Bx \eand \exists y(
\item %Qualquer doce com chocolate é melhor que qualquer doce sem ele.
$\forall x[Cx \eif \forall y(\enot Cy \eif Bxy)]$
\item %Qualquer doce com chocolate e maçapão é melhor que qualquer doce sem nenhum dos dois.
$\forall x\bigl((Cx \eand Mx) \eif \forall y[(\enot Cy \eand \enot My) \eif Bxy]\bigr)$
\end{earg}


\solutionsection{ch.LQ}{pr.QLballet}
\begin{earg}
\item %Todos os filhos de Patrick são bailarinos de balé.
$\forall x(Cxp \eif Dx)$
\item %Jane é filha de Patrick.
$Cjp \eand Fj$
\item %Patrick tem uma filha.
$\exists x(Cxp \eand Fx)$
\item %Jane é filha única.
$\enot\exists x Sxj$
\item %Todas as filhas de Patrick dançam balé.
$\forall x\bigl[(Cxp \eand Fx)\eif Dx\bigr]$
\item %Patrick não tem filhos.
$\enot\exists x(Cxp \eand Mx)$
\item %Jane é sobrinha de Elmer.
$\exists x(Cjx \eand Sxe \eand Fj)$
\item %Patrick é irmão de Elmer.
$Spe \eand Mp$
\item %Os irmãos de Patrick não têm filhos.
$\forall x\bigl[(Sxp \eand Mx) \eif \enot\exists y Cyx\bigr]$
\item %Jane é tia.
$\exists x(Sxj \eand \exists y Cyx \eand Fj)$
\item %Todos que dançam no balé têm uma irmã que também dança no balé.
$\forall x\bigl[Dx \eif \exists y(Sxy \eand Fy \eand Dy)\bigr]$
\item %Todo homem que dança no balé é filho de alguém que dança no balé.
$\forall x\bigl[(Mx \eand Dx) \eif \exists y(Cxy \eand Dy)\bigr]$
\end{earg}

\solutionsection{ch.LQ}{pr.QLcards}
\begin{earg}
\item %Todos os paus são cartas pretas.
$\forall x(Cx \eif Bx)$
\item %Não há curingas.
$\enot\exists x Wx$
\item %Há pelo menos dois paus.
$\exists x \exists y(Cx \eand Cy \eand x\neq y)$
\item %Há mais de um valete de um olho só.
$\exists x \exists y(Jx \eand Ox \eand Jy \eand Oy \eand x\neq y)$
\item %Há no máximo dois valetes de um olho só.
$\forall x\forall y\forall z\bigl[(Jx \eand Ox \eand Jy \eand Oy \eand Jz \eand Oz)\eif(x=y \eor x=z \eor y=z)\bigr]$
\item %Há dois valetes pretos.
$\exists x\exists y\bigl(Jx \eand Bx \eand Jy \eand By \eand x \neq y\eand \forall z[(Jz \eand Bz) \eif (x=z \eor y=z)]\bigr)$
\item %Há quatro duques.
$\exists x_1\exists x_2\exists x_3\exists x_4\bigl[Dx_1 \eand Dx_2 \eand Dx_3 \eand Dx_4 \eand x_1 \neq x_2 \eand x_1 \neq x_3 \eand x_1 \neq x_4 \eand x_2 \neq x_3 \eand x_2 \neq x_4 \eand x_3 \neq x_4  \eand \enot\exists y(Dy \eand y\neq x_1 \eand y\neq x_2 \eand y\neq x_3 \eand y\neq x_4)\bigr]$
\item %O duque de paus é uma carta preta.
$\exists x\bigl(Dx \eand Cx \eand \forall y[(Dy\eand Cy) \eif x=y] \eand Bx\bigr)$
\item %Valetes de um olho só e o homem com o machado são curingas.
$\forall x\bigl[(Ox \eand Jx) \eif Wx\bigr] \eand \exists x\bigl[Mx \eand \forall y(My \eif x=y) \eand Wx\bigr]$
\item %Se o duque de paus é curinga, então há exatamente um curinga.
$\exists x\bigl(Dx \eand Cx \eand \forall y[(Dy\eand Cy) \eif x=y] \eand Wx\bigr)\eif \exists x\forall y(Wx \eiff x=y)$
\item %O homem com o machado não é um valete.
escopo amplo: $\enot \exists x\bigl[Mx \eand \forall y(My \eif x=y) \eand Jx\bigr]$\\
escopo restrito: $\exists x\bigl[Mx \eand \forall y(My \eif x=y) \eand \enot Jx\bigr]$
\item %O duque de paus não é o homem com o machado.
escopo amplo: $\enot \exists x\exists z\bigl[Dx \eand Cx \eand Mz \eand \forall y[(Dy\eand Cy) \eif x=y]  \eand \forall y[(My \eif z=y) \eand x=z]\bigr]$\\
escopo restrito: $\exists x\exists z\bigl[Dx \eand Cx \eand Mz \eand \forall y[(Dy\eand Cy) \eif x=y]  \eand \forall y[(My \eif z=y) \eand x\neq z]\bigr]$
\end{earg}





\solutionsection{ch.semantics}{pr.TorF1}
%\item $Bc$
\noSeq
%\item $Ac \eiff \enot Nc$
\nextSeq
%\item $Nc \eif (Ac \eor Bc)$
\nextSeq
%\item $\forall x Ax$
\nextSeq
%\item $\forall x \enot Bx$
\noSeq
%\item $\exists x(Ax \eand Bx)$
\nextSeq
%\item $\exists x(Ax \eif Nx)$
\noSeq
%\item $\forall x(Nx \eor \enot Nx)$
\nextSeq
%\item $\exists x Bx \eif \forall x Ax$
\lastSeq
são verdadeiras no modelo.

\solutionsection{ch.semantics}{pr.TorF2}
\noSeq%\item $\exists x(Rxm \eand Rmx)$
\noSeq%\item $\forall x(Rxm \eor Rmx)$
\noSeq%\item $\forall x(Hx \eiff Wx)$
\nextSeq%\item $\forall x(Rxm \eif Wx)$
\nextSeq%\item $\forall x\bigl[Wx \eif(Hx \eand Wx)\bigr]$
\noSeq%\item $\exists x Rxx$
\lastSeq%\item $\exists x\exists y Rxy$
\noSeq%\item $\forall x \forall y Rxy$
\noSeq%\item $\forall x \forall y (Rxy \eor Ryx)$
\noSeq%\item $\forallx \forall y \forall z\bigl[(Rxy \eand Ryz) \eif Rxz\bigr]$
são verdadeiras no modelo.

\solutionsection{ch.semantics}{pr.InterpretationToModel}
\begin{partialmodel}
UD & \{10,11,12,13\}\\
\extension{O} & \{11,13\}\\
\extension{S} & $\emptyset$\\
\extension{T} & \{10,11,12,13\}\\
\extension{U} & \{13\}\\
\extension{N} & \{\ntuple{11,10},\ntuple{12,11},\ntuple{13,12}\}\\
\end{partialmodel}



\solutionsection{ch.semantics}{pr.Contingent}
\begin{earg}
\item %$Da \eand Db$
	A sentença é verdadeira neste modelo:
	\begin{partialmodel}
		UD & \{Stan\}\\
		\extension{D} & \{Stan\}\\
		\referent{a} & Stan\\
		\referent{b} & Stan
	\end{partialmodel}
	E é falsa neste modelo:
	\begin{partialmodel}
		UD & \{Stan\}\\
		\extension{D} & $\emptyset$\\
		\referent{a} & Stan\\
		\referent{b} & Stan
	\end{partialmodel}
\item %$\exists x Txh$
	A sentença é verdadeira neste modelo:
	\begin{partialmodel}
		UD & \{Stan\}\\
		\extension{T} & \{\ntuple{Stan, Stan}\}\\
		\referent{h} & Stan
	\end{partialmodel}
	E é falsa neste modelo:
	\begin{partialmodel}
		UD & \{Stan\}\\\
		\extension{T} & $\emptyset$\\
		\referent{h} & Stan
	\end{partialmodel}
\item %$Pm \eand \enot\forall x Px$
	A sentença é verdadeira neste modelo:
	\begin{partialmodel}
		UD & \{Stan, Ollie\}\\
		\extension{P} & \{Stan\}\\
		\referent{m} & Stan
	\end{partialmodel}
	E é falsa neste modelo:
	\begin{partialmodel}
		UD & \{Stan\}\\
		\extension{P} & $\emptyset$\\
		\referent{m} & Stan
	\end{partialmodel}
\end{earg}



\solutionsection{ch.semantics}{pr.NotEquiv}
Há muitas respostas corretas possíveis. Aqui estão algumas:
\begin{earg}
\item %$Ja$, $Ka$
	Tornando a primeira sentença verdadeira e a segunda falsa:
	\begin{partialmodel}
		UD & \{alfa\}\\
		\extension{J} & \{alfa\}\\
		\extension{K} & $\emptyset$\\
		\referent{a} & alfa
	\end{partialmodel}
\item %$\exists x Jx$, $Jm$
	Tornando a primeira sentença verdadeira e a segunda falsa:
	\begin{partialmodel}
		UD & \{alfa, ômega\}\\
		\extension{J} & \{alfa\}\\\
		\referent{m} & ômega
	\end{partialmodel}
\item %$\forall x Rxx$, $\exists x Rxx$
	Tornando a primeira sentença falsa e a segunda verdadeira:
	\begin{partialmodel}
		UD & \{alfa, ômega\}\\
		\extension{R} & \{\ntuple{alfa,alfa}\}
	\end{partialmodel}
\item %$\exists x Px \eif Qc$, $\exists x (Px \eif Qc)$
	Tornando a primeira sentença falsa e a segunda verdadeira:
	\begin{partialmodel}
		UD & \{alfa, ômega\}\\
		\extension{P} & \{alfa\}\\
		\extension{Q} & $\emptyset$\\
		\referent{c} & alfa
	\end{partialmodel}
\item %$\forall x(Px \eif \enot Qx)$, $\exists x(Px \eand \enot Qx)$
	Tornando a primeira sentença verdadeira e a segunda falsa:
	\begin{partialmodel}
		UD & \{iota\}\\
		\extension{P} & $\emptyset$\\
		\extension{Q} & $\emptyset$
	\end{partialmodel}
\item %$\exists x(Px \eand Qx)$, $\exists x(Px \eif Qx)$
	Tornando a primeira sentença falsa e a segunda verdadeira:
	\begin{partialmodel}
		UD & \{iota\}\\
		\extension{P} & $\emptyset$\\
		\extension{Q} & \{iota\}
	\end{partialmodel}
\item %$\forall x(Px\eif Qx)$, $\forall x(Px \eand Qx)$
	Tornando a primeira sentença verdadeira e a segunda falsa:
	\begin{partialmodel}
		UD & \{iota\}\\
		\extension{P} & $\emptyset$\\
		\extension{Q} & \{iota\}
	\end{partialmodel}
\item %$\forall x\exists y Rxy$, $\exists x\forall y Rxy$
	Tornando a primeira sentença verdadeira e a segunda falsa:
	\begin{partialmodel}
		UD & \{alfa, ômega\}\\
		\extension{R} & \{\ntuple{alfa, ômega}, \ntuple{ômega, alfa}\}
	\end{partialmodel}
\item %$\forall x\exists y Rxy$, $\forall x\exists y Ryx$
	Tornando a primeira sentença falsa e a segunda verdadeira:
	\begin{partialmodel}
		UD & \{alfa, ômega\}\\
		\extension{R} & \{\ntuple{alfa, alfa}, \ntuple{alfa, ômega}\}
	\end{partialmodel}
\end{earg}

\solutionsection{ch.semantics}{pr.IdentityModels}
\begin{earg}
\item %Mostre que $\{{\enot}Raa, \forall x (x=a \eor Rxa)\}$ é consistente.
Há muitas respostas possíveis. Aqui está uma:
\begin{partialmodel}
UD & \{Harry, Sally\}\\
\extension{R} &\{\ntuple{Sally, Harry}\}\\
\referent{a} & Harry
\end{partialmodel}
\item %Mostre que $\{\forall x\forall y\forall z(x=y \eor y=z \eor x=z), \exists x\exists y\ x\neq y\}$ é consistente.
Não há predicados ou constantes, então só precisamos dar um UD.
Qualquer UD com 2 membros servirá.
\item %Mostre que $\{\forall x\forall y\ x=y, \exists x\ x \neq a\}$ é inconsistente.
Precisamos mostrar que é impossível construir um modelo no qual ambos são verdadeiros. Suponha que $\exists x\ x \neq a$ é verdadeira em um modelo. Há algo no universo de discurso que \emph{não} é o referente de $a$. Então há pelo menos duas coisas no universo de discurso: \referent{a} e essa outra coisa. Chame essa outra coisa de $\beta$--- sabemos $a \neq \beta$. Mas se $a \neq \beta$, então $\forall x\forall y\ x=y$ é falsa. Então a primeira sentença deve ser falsa se a segunda sentença é verdadeira. Como tal, não há modelo no qual ambas são verdadeiras. Portanto, são inconsistentes.
\end{earg}

\solutionsection{ch.semantics}{pr.SemanticsEssay}
\begin{earg}
\stepcounter{eargnum}
\item Não, não faria diferença. A satisfação de uma fórmula com uma ou mais variáveis livres depende do que a atribuição de variável faz para essas variáveis. Porém, como uma sentença não tem variáveis livres, sua satisfação não depende da atribuição de variável. Então uma sentença que é satisfeita por \emph{alguma} atribuição de variável é satisfeita por \emph{toda} outra atribuição de variável também.
\end{earg}


\solutionsection{ch.proofs}{pr.justifySLproof}

\begin{multicols}{2}
\begin{proof}
\hypo{1}{W \eif \enot B}
\hypo{2}{A \eand W}
\hypo{2b}{B \eor (J \eand K)}
\have{3}{W}\ae{2}
\have{4}{\enot B} \ce{1,3}
\have{5}{J \eand K} \oe{2b,4}
\have{6}{K} \ae{5}
\end{proof}

\begin{proof}
\hypo{1}{L \eiff \enot O}
\hypo{2}{L \eor \enot O}
\open
	\hypo{a1}{\enot L}
	\have{a2}{\enot O}\oe{2,a1}
	\have{a3}{L}\be{1,a2}
	\have{a4}{\enot L}\by{R}{a1}
\close
\have{3}{L}\ne{a1-a4}
\end{proof}

\begin{proof}
\hypo{1}{Z \eif (C \eand \enot N)}
\hypo{2}{\enot Z \eif (N \eand \enot C)}
\open
	\hypo{a1}{\enot(N \eor  C)}
	\have{a2}{\enot N \eand \enot C} \by{DeM}{a1}
	\open
		\hypo{b1}{Z} 
		\have{b2}{C \eand \enot N}\ce{1,b1}
		\have{b3}{C}\ae{b2}
		\have{b4}{\enot C}\ae{a2}
	\close
	\have{a3}{\enot Z}\ni{b1-b4}
	\have{a4}{N \eand \enot C}\ce{2,a3}
	\have{a5}{N}\ae{a4}
	\have{a6}{\enot N}\ae{a2}
\close
\have{3}{N \eor C}\ne{a1-a6}
\end{proof}
\end{multicols}

\solutionsection{ch.proofs}{pr.solvedSLproofs}
%Dê uma prova para cada argumento em SL.
\begin{earg}
\item%$K\eand L$, \therefore $K\eiff L$
\begin{solutioninlist}
\begin{proof}
	\hypo{a1}{K\eand L} \want{K\eiff L}
	\open
		\hypo{b1}{K} \want{L}
		\have{b2}{L} \ae{a1}
	\close
	\open
		\hypo{c1}{L} \want{K}
		\have{c2}{K} \ae{a1}
	\close
	\have{d1}{K \eiff L} \bi{b1-b2,c1-c2}
\end{proof}
\end{solutioninlist}
\item%$A\eif (B\eif C)$, \therefore $(A\eand B)\eif C$
\begin{solutioninlist}
\begin{proof}
	\hypo{a1}{A\eif (B\eif C)} \want{(A\eand B)\eif C}
	\open
		\hypo{b1}{A\eand B} \want{C}
		\have{b2}{A} \ae{b1}
		\have{b3}{B\eif C} \ce{a1,b2}
		\have{b4}{B} \ae{b1}
		\have{b5}{C} \ce{b3, b4}
	\close
	\have{c}{(A\eand B)\eif C} \ci{b1-b5}
\end{proof}
\end{solutioninlist}
\item%$P \eand (Q\eor R)$, $P\eif \enot R$, \therefore $Q\eor E$
\begin{solutioninlist}
\begin{proof}
	\hypo{a1}{P \eand (Q\eor R)}
	\hypo{a2}{P\eif \enot R} \want{Q\eor E}
	\have{a3}{P} \andE{a1}
	\have{a4}{\enot R} \ifE{a2,a3}
	\have{a5}{Q\eor R} \andE{a1}
	\have{a6}{Q} \orE{a5,a4}
	\have{c}{Q\eor E} \orI{a6}
\end{proof}
\end{solutioninlist}
\item%$(C\eand D)\eor E$, \therefore $E\eor D$
\begin{solutioninlist}
\begin{proof}
	\hypo{a1}{(C\eand D)\eor E} \want{E\eor D}
	\open
		\hypo{b1}{\enot E} \want{D}
		\have{b2}{C\eand D} \oe{a1,b1}
		\have{b3}{D} \ae{b2}
	\close
	\have{c1}{\enot E\eif D} \ci{b1-b3}
	\have{c2}{E\eor D} \by{MC}{c1}
\end{proof}
\end{solutioninlist}
\item%$\enot F\eif G$, $F\eif H$, \therefore $G\eor H$
\begin{solutioninlist}
\begin{proof}
	\hypo{a1}{\enot F\eif G}
	\hypo{a2}{F\eif H} \want{G\eor H}
	\open
		\hypo{b1}{\enot G} \want{H}
		\have{b2}{\enot\enot F} \by{MT}{a1,b1}
		\have{b3}{F} \by{DN}{b2}
		\have{b4}{H} \ce{a2,b3}
	\close
	\have{c1}{\enot G \eif H} \ci{b1-b4}
	\have{c2}{G\eor H} \by{MC}{c1}
\end{proof}
\end{solutioninlist}
\item%$(X\eand Y)\eor(X\eand Z)$, $\enot(X\eand D)$, $D\eor M$ \therefore $M$
\begin{solutioninlist}
\begin{proof}
	\hypo{a1}{(X\eand Y)\eor(X\eand Z)}
	\hypo{a2}{\enot(X\eand D)}
	\hypo{a3}{D\eor M} \want{M}
	\open
		\hypo{b1}{\enot X} \by{para reductio}{}
		\have{b2}{\enot X\eor \enot Y} \oi{b1}
		\have{b3}{\enot (X \eand Y)} \by{DeM}{b2}
		\have{b4}{X\eand Z} \oe{a1,b3}
		\have{b5}{X} \ae{b4}
		\have{b6}{\enot X} \by{R}{b1}
	\close
	\have{c}{X} \ne{b1-b6}
	\open
		\hypo{d1}{\enot M} \by{para reductio}{}
		\have{d2}{D} \oe{a3,d1}
		\have{d3}{X\eand D} \ai{c,d2}
		\have{d4}{\enot(X\eand D)} \by{R}{a2}
	\close
	\have{e}{M} \ne{d1-d4}
\end{proof}
\end{solutioninlist}
\end{earg}


\solutionsection{ch.proofs}{pr.subinstanceQL}
\begin{earg}
\item $Rca$, $Rcb$, $Rcc$ e $Rcd$ são instâncias de substituição de $\forall x Rcx$.
\item Das expressões listadas, apenas $\forall y Lby$ é uma instância de substituição de $\exists x\forall y Lxy$.
\end{earg}


\solutionsection{ch.proofs}{pr.justifyQLproof}

\begin{multicols}{2}
%$\{\forall x(\exists y)(Rxy \eor Ryx),\forall x\enot Rmx\}\vdash\exists xRxm$
\begin{proof}
\hypo{p1}{\forall x\exists y(Rxy \eor Ryx)}
\hypo{p2}{\forall x\enot Rmx}
\have{3}{\exists y(Rmy \eor Rym)}\Ae{p1}
	\open
		\hypo{a1}{Rma \eor Ram}
		\have{a2}{\enot Rma}\Ae{p2}
		\have{a3}{Ram}\oe{a1,a2}
		\have{a4}{\exists x Rxm}\Ei{a3}
	\close
\have{n}{\exists x Rxm} \Ee{3,a1-a4}
\end{proof}

%$\{\forall x(\exists yLxy \eif \forall zLzx), Lab\} \vdash \forall xLxx$
\begin{proof}
\hypo{1}{\forall x(\exists yLxy \eif \forall zLzx)}
\hypo{2}{Lab}
\have{3}{\exists y Lay \eif \forall zLza}\Ae{1}
\have{4}{\exists y Lay} \Ei{2}
\have{5}{\forall z Lza} \ce{3,4}
\have{6}{Lca}\Ae{5}
\have{7}{\exists y Lcy \eif \forall zLzc}\Ae{1}
\have{8}{\exists y Lcy}\Ei{6}
\have{9}{\forall z Lzc}\ce{7,8}
\have{10}{Lcc}\Ae{9}
\have{11}{\forall x Lxx}\Ai{10}
\end{proof}

% $\{\forall x(Jx \eif Kx), \exists x\forall y Lxy, \forall x Jx\} \vdash \exists x(Kx \eand Lxx)$
\begin{proof}
\hypo{a}{\forall x(Jx \eif Kx)}
\hypo{b}{\exists x\forall y Lxy}
\hypo{c}{\forall x Jx}
\open
	\hypo{2}{\forall y Lay}
	\have{d}{Ja}\Ae{c}
	\have{e}{Ja \eif Ka}\Ae{a}
	\have{f}{Ka}\ce{e,d}
	\have{3}{Laa}\Ae{2}
	\have{4}{Ka \eand Laa}\ai{f,3}
	\have{5}{\exists x(Kx \eand Lxx)}\Ei{4}
\close
\have{j}{\exists x(Kx \eand Lxx)}\Ee{b,2-5}
\end{proof}


%$\vdash \exists x Mx \eor \forall x\enot Mx$
\begin{proof}
	\open
		\hypo{p1}{\enot (\exists x Mx \eor \forall x\enot Mx)}
		\have{p2}{\enot \exists x Mx \eand \enot \forall x\enot Mx} \by{DeM}{p1}
		\have{p3}{\enot \exists x Mx}\ae{p2}
		\have{p4}{\forall x\enot Mx}\by{QN}{p3}
		\have{p5}{\enot \forall x\enot Mx}\ae{p2}
	\close
\have{n}{\exists x Mx \eor \forall x\enot Mx} \ne{p1-p5}
\end{proof}
\end{multicols}


\solutionsection{ch.proofs}{pr.someQLproofs}

\begin{earg}
\item%$\vdash \forall x Fx \eor \enot \forall x Fx$
\begin{solutioninlist}
\begin{proof}
	\open
		\hypo{p1}{\enot(\forall x Fx \eor \enot\forall x Fx)} \by{para reductio}{}
		\have{s1}{\enot\forall x Fx \eand \enot\enot\forall x Fx} \by{DeM}{p1}
		\have{s2}{\enot \forall x Fx}\ae{s1}
		\have{s3}{\enot\enot\forall x Fx}\ae{s1}
	\close
	\have{c}{\forall x Fx \eor \enot\forall x Fx} \ne{p1-s3}
\end{proof}
\end{solutioninlist}
\item %$\{\forall x(Mx \eiff Nx), Ma\eand\exists x Rxa\}\vdash \exists x Nx$
\begin{solutioninlist}
\begin{proof}
	\hypo{p1}{\forall x(Mx \eiff Nx)}
	\hypo{p2}{Ma \eand \exists x Rxa} \by{quer $\exists x Nx$}{}
	\have{a1}{Ma \eiff Na} \Ae{p1}
	\have{a2}{Ma} \ae{p2}
	\have{a3}{Na} \be{a1,a2}
	\have{c}{\exists x Nx} \Ei{a3}
\end{proof}
\end{solutioninlist}
\item %$\{\forall x(\enot Mx \eor Ljx), \forall x(Bx\eif Ljx), \forall x(Mx\eor Bx)\}\vdash \forall xLjx$
\begin{solutioninlist}
\begin{proof}
	\hypo{a1}{\forall x(\enot Mx \eor Ljx)}
	\hypo{a2}{\forall x(Bx \eif Ljx)}
	\hypo{a3}{\forall x(Mx \eor Bx)} \by{quer $\forall x Ljx$}{}
	\have{a4}{\enot Ma \eor Lja} \Ae{a1}
	\have{a5}{Ma \eif Lja} \by{MC}{a4}
	\have{a6}{Ba \eif Lja} \Ae{a2}
	\have{a7}{Ma \eor Ba} \Ae{a3}
	\have{a8}{Lja}  \by{DIL}{a7, a5, a6}
	\have{a9}{\forall x Ljx} \Ai{a8}
\end{proof}
\end{solutioninlist}
\item %$\forall x(Cx \eand Dt)\vdash \forall xCx \eand Dt$
\begin{solutioninlist}
\begin{proof}
	\hypo{a1}{\forall x(Cx \eand Dt)} \by{quer $\forall x Cx \eand Dt$}{}
	\have{a2}{Ca \eand Dt} \Ae{a1}
	\have{a3}{Ca} \ae{a2}
	\have{a4}{\forall x Cx} \Ai{a3}
	\have{a5}{Dt} \ae{a2}
	\have{a6}{\forall x Cx \eand Dt} \ai{a4, a5}
\end{proof}
\end{solutioninlist}
\item %$\exists x(Cx \eor Dt)\vdash \exists x Cx \eor Dt$
\begin{solutioninlist}
\begin{proof}
	\hypo{a1}{\exists x(Cx \eor Dt)} \by{quer $\exists x Cx \eor Dt$}{}
	\open
		\hypo{a2}{Ca \eor Dt} \by{para {$\exists$}E}{}
		\open
			\hypo{a3}{\enot(\exists x Cx \eor Dt)} \by{para reductio}{}
			\have{a4}{\enot\exists x Cx \eand \enot Dt} \by{DeM}{a3}
			\have{a5}{\enot Dt} \ae{a4}
			\have{a6}{Ca} \oe{a2,a5}
			\have{a7}{\exists x Cx} \Ei{a6}
			\have{a8}{\enot\exists x Cx} \ae{a4}
		\close
		\have{a9}{\exists x Cx \eor Dt} \ne{a3-a8}
	\close
	\have{a10}{\exists x Cx \eor Dt} \Ee{a1,a2-a9}
\end{proof}
\end{solutioninlist}
\end{earg}



\solutionsection{ch.proofs}{pr.likes}
Quanto à tradução deste argumento, veja p.~\pageref{likes2}.

\begin{proof}
\hypo{1}{\exists x\forall y[\forall z(Lxz \eif Lyz) \eif Lxy]}
\open
	\hypo{a}{\forall y[\forall z(Laz \eif Lyz) \eif Lay]}
	\have{b}{\forall z(Laz \eif Laz) \eif Laa} \Ae{a}
	\open
		\hypo{c1}{\enot \exists x Lxx} \by{para reductio}{}
		\have{c2}{\forall x\enot Lxx} \by{QN}{c1}
		\have{c3}{\enot Laa} \Ae{c2}
		\have{c4}{\enot \forall z(Laz \eif Laz)} \by{MT}{c2,c3}
		\open
			\hypo{d1}{Lab}
			\have{d2}{Lab} \by{R}{d1}
		\close
		\have{c5}{Lab \eif Lab} \ci{d1--d2}
		\have{c6}{\forall z(Laz \eif Laz)} \Ai{c5}
		\have{cn}{\enot \forall z(Laz \eif Laz)} \by{R}{c4}
	\close
	\have{h}{\exists x Lxx} \ne{c1--cn}
\close
\have{n}{\exists x Lxx} \Ee{1, a--h}
\end{proof}


\solutionsection{ch.proofs}{pr.QLequivornot}
%\item $\forall x Px \eif Qc$, $\forall x (Px \eif Qc)$
\noSeq
%\item $\forall x Px \eand Qc$, $\forall x (Px \eand Qc)$
\nextSeq
%\item $Qc \eor \exists x Qx$, $\exists x (Qc \eor Qx)$
\nextSeq
%\item $\forall x\forall y \forall z Bxyz$, $\forall x Bxxx$
\noSeq
%\item $\forall x\forall y Dxy$, $\forall y\forall x Dxy$
\lastSeq
%\item $\exists x\forall y Dxy$, $\forall y\exists x Dxy$
\noSeq
são logicamente equivalentes.

\solutionsection{ch.proofs}{pr.QLvalidornot}
\noSeq%\item $\forall x\exists y Rxy$, \therefore\ $\exists y\forall x Rxy$
\nextSeq%\item $\exists y\forall x Rxy$, \therefore\ $\forall x\exists y Rxy$
\noSeq%\item $\exists x(Px \eand \enot Qx)$, \therefore\ $\forall x(Px \eif \enot Qx)$
\nextSeq%\item $\forall x(Sx \eif Ta)$, $Sd$, \therefore\ $Ta$
\nextSeq%\item $\forall x(Ax\eif Bx)$, $\forall x(Bx \eif Cx)$, \therefore\ $\forall x(Ax \eif Cx)$
\noSeq%\item $\exists x(Dx \eor Ex)$, $\forall x(Dx \eif Fx)$, \therefore\ $\exists x(Dx \eand Fx)$
\nextSeq%\item $\forall x\forall y(Rxy \eor Ryx)$, \therefore\ $Rjj$
\noSeq%\item $\exists x\exists y(Rxy \eor Ryx)$, \therefore\ $Rjj$
\noSeq%\item $\forall x Px \eif \forall x Qx$, $\exists x \enot Px$, \therefore\ $\exists x \enot Qx$
\lastSeq%\item $\exists x Mx \eif \exists x Nx$, $\enot \exists x Nx$, \therefore\ $\forall x \enot Mx$
são válidos. Aqui estão respostas completas para alguns deles:
\begin{earg}
\item %$\forall x\exists y Rxy$, \therefore\ $\exists y\forall x Rxy$
	\begin{solutioninlist}
	\begin{partialmodel}
		UD & \{mocha, freddo\}\\
		\extension{R} & \{\ntuple{mocha, freddo}, \ntuple{freddo, mocha}\}
	\end{partialmodel}
	\end{solutioninlist}
\item %$\exists y\forall x Rxy$, \therefore\ $\forall x\exists y Rxy$
	\begin{solutioninlist}
	\begin{proof}
		\hypo{p}{\exists y\forall x Rxy} \by{quer $\forall x\exists y Rxy$}{}
		\open
			\hypo{ass}{\forall x Rxa}
			\have{R}{Rba} \Ae{ass}
			\have{ER}{\exists y Rby} \Ei{R}
			\have{AER}{\forall x \exists y Rxy} \Ai{ER}
		\close
		\have{c}{\forall x\exists y Rxy} \Ee{p, ass-AER}
	\end{proof}
	\end{solutioninlist}
\end{earg}
