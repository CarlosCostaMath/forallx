%!TEX root = forallx.tex
\addcontentsline{toc}{chapter}{Referência Rápida}
\pagestyle{plain}
{\LARGE \bf Referência Rápida}

%\section*{Tabelas Verdade Características}
\label{app.CharacteristicTTs}

\hfill
\begin{tabular}{c|c}
\script{A} & \enot\script{A}\\
\hline
V & F\\
F & V 
\end{tabular}
\hfill
\begin{tabular}{c|c|c|c|c|c}
\script{A} & \script{B} & \script{A}\eand\script{B} & \script{A}\eor\script{B} & \script{A}\eif\script{B} & \script{A}\eiff\script{B}\\
\hline
V & V & V & V & V & V\\
V & F & F & V & F & F\\
F & V & F & V & V & F\\
F & F & F & F & V & V
\end{tabular}
\hfill

\vfill


\hfill
\begin{tabular}{c|c}
\script{A} & \enot\script{A}\\
\hline
1 & 0\\
0 & 1 
\end{tabular}
\hfill
\begin{tabular}{c|c|c|c|c|c}
\script{A} & \script{B} & \script{A}\eand\script{B} & \script{A}\eor\script{B} & \script{A}\eif\script{B} & \script{A}\eiff\script{B}\\
\hline
1 & 1 & 1 & 1 & 1 & 1\\
1 & 0 & 0 & 1 & 0 & 0\\
0 & 1 & 0 & 1 & 1 & 0\\
0 & 0 & 0 & 0 & 1 & 1
\end{tabular}
\hfill

\vfill


\section*{Simbolização}
\begin{center}
\label{app.symbolization}
\begin{tabular*}{\textwidth}{rl}
\multicolumn{2}{c}{\textsc{Conectivos Sentenciais} (capítulo \ref{ch.SL})}\\ \\
Não é o caso que $P$. & $\enot P$\\
Ou $P$, ou $Q$. & $(P \eor Q)$\\
Nem $P$, nem $Q$. & $\enot(P \eor Q)$\ ou \ $(\enot P \eand \enot Q)$\\
Tanto $P$, quanto $Q$. & $(P \eand Q)$\\
Se $P$, então $Q$. & $(P \eif Q)$\\
$P$ somente se $Q$. & $(P \eif Q)$\\
$P$ se e somente se $Q$. & $(P \eiff Q)$\\
A menos que $P$, $Q$. $P$ a menos que $Q$. & $(P \eor Q)$\\
\\
\multicolumn{2}{c}{\label{SymbolizingPredicates}\textsc{Predicados} (capítulo \ref{ch.QL})}\\ \\
Todo $F$ é $G$. & $\forall x(Fx \eif Gx)$\\
Algum $F$ é $G$. & $\exists x(Fx \eand Gx)$\\
Nem todo $F$ é $G$. & $\enot\forall x(Fx \eif Gx)$\ ou\ $\exists x(Fx \eand \enot Gx)$\\
Nenhum $F$ é $G$. & $\forall x(Fx \eif\enot Gx)$\ ou\ $\enot\exists x(Fx \eand Gx)$\\
\\
\multicolumn{2}{c}{\textsc{Identidade} (seção \ref{sec.identity})}\\ \\
Apenas $j$ é $G$. & $\forall x(Gx \eiff x=j)$\\
Tudo exceto $j$ é $G$. & $\forall x(x \neq j \eif Gx)$\\
%$j$ é mais $R$ que qualquer outro. & $\forall x(x\neq j \eif Rjx)$\\
O $F$ é $G$. & $\exists x(Fx \eand \forall y(Fy \eif x=y) \eand Gx)$\\
\multicolumn{2}{l}{`O F não é G' pode ser traduzido de duas formas:} \\
Não é o caso que o F é G. (amplo)& $\enot\exists x(Fx \eand \forall y(Fy \eif x=y) \eand Gx)$\\
O $F$ é não-$G$. (restrito) & $\exists x(Fx \eand \forall y(Fy \eif x=y) \eand \enot Gx)$
\end{tabular*}
\end{center}

% BEGIN: symbolizing cardinality

\newpage
\section*{Usando identidade para simbolizar quantidades}

\subsection*{Existem pelo menos \blank\ $F$s.}
\label{summary.atleast}

\begin{ekey}
\item[um] $\exists xFx$
\item[dois] $\exists x_1\exists x_2(Fx_1 \eand Fx_2 \eand x_1 \neq x_2)$
\item[três] $\exists x_1\exists x_2\exists x_3(Fx_1 \eand Fx_2 \eand Fx_3 \eand x_1 \neq x_2 \eand x_1 \neq x_3 \eand x_2 \neq x_3)$
\item[quatro] $\exists x_1\exists x_2\exists x_3\exists x_4 (Fx_1 \eand Fx_2 \eand Fx_3 \eand Fx_4 \eand x_1 \neq x_2 \eand x_1 \neq x_3 \eand x_1 \neq x_4 \eand x_2 \neq x_3 \eand x_2 \neq x_4 \eand x_3 \neq x_4)$
\item[n] $\exists x_1\cdots\exists x_n(Fx_1 \eand\cdots\eand Fx_n \eand x_1 \neq x_2 \eand\cdots\eand x_{n-1}\neq x_n)$ 
\end{ekey}

\subsection*{Existem no máximo \blank\ $F$s.}
\label{summary.atmost}

Uma forma de dizer `no máximo $n$ coisas são $F$' é colocar um sinal de negação antes de uma das simbolizações acima e dizer $\enot$`pelo menos $n+1$ coisas são $F$.' Equivalentemente:
\begin{ekey}
\item[um] $\forall x_1\forall x_2\bigl[(Fx_1 \eand Fx_2) \eif x_1=x_2\bigr]$
\item[dois] $\forall x_1\forall x_2\forall x_3\bigl[(Fx_1 \eand Fx_2 \eand Fx_3) \eif (x_1=x_2 \eor x_1=x_3 \eor x_2=x_3)\bigr]$
\item[três] $\forall x_1\forall x_2\forall x_3\forall x_4\bigl[(Fx_1 \eand Fx_2 \eand Fx_3 \eand Fx_4) \eif (x_1=x_2 \eor x_1=x_3 \eor x_1=x_4 \eor x_2=x_3 \eor x_2=x_4 \eor x_3=x_4)\bigr]$
\item[n]$\forall x_1\cdots\forall x_{n+1}
\bigl[(Fx_1\eand \cdots \eand Fx_{n+1}) \eif (x_1=x_2 \eor \cdots \eor x_n=x_{n+1})\bigr]$ 
\end{ekey}

\subsection*{Existem exatamente \blank\ $F$s.}
\label{summary.exactly}

Uma forma de dizer `exatamente $n$ coisas são $F$' é conjugar duas das simbolizações acima e dizer `pelo menos $n$ coisas são $F$' \eand\ `no máximo $n$ coisas são $F$.' As seguintes fórmulas equivalentes são mais curtas:
\begin{ekey}
\item[zero] $\forall x\enot Fx$
\item[um] $\exists x\bigl[Fx \eand \enot\exists y(Fy \eand x\neq y)\bigr]$
\item[dois] $\exists x_1\exists x_2\bigl[Fx_1 \eand Fx_2 \eand x_1 \neq x_2 \eand \enot\exists y\bigl(Fy \eand y\neq x_1 \eand y \neq x_2\bigr) \bigr]$
\item[três] $\exists x_1\exists x_2\exists x_3\bigl[Fx_1 \eand Fx_2 \eand Fx_3 \eand x_1 \neq x_2 \eand x_1 \neq x_3 \eand x_2 \neq x_3 \eand\\
\enot\exists y(Fy \eand y \neq x_1 \eand y \neq x_2 \eand y\neq x_3) \bigr]$
\item[n] $\exists x_1\cdots\exists x_n\bigl[Fx_1 \eand\cdots\eand Fx_n  \eand x_1 \neq x_2 \eand\cdots\eand x_{n-1}\neq x_n \eand\\
 \enot\exists y(Fy \eand y\neq x_1 \eand \cdots \eand y\neq x_n)\bigr]$ 
%\item[um] $\exists x\forall y\bigl[Fx \eand (Fy \eif y = x)\bigr]$
%\item[dois] $\exists x\exists y\forall z\Bigl(Fx \eand Fy \eand \bigl[Fz \eif (z=x \eor z=y)\bigr] \eand x \neq y\Bigr)$
%\item[três] $\exists x_1\exists x_2\exists x_3\forall y\Bigl(Fx_1 \eand Fx_2 \eand Fx_3 \eand [Fy \eif (y=x_1 \eor y=x_2 \eor y=x_3)] \eand x_1 \neq x_2 \eand x_1 \neq x_3 \eand x_2 \neq x_3\Bigr)$
%\item[n] $\exists x_1\cdots\exists x_n\forall y\Bigl(Fx_1 \eand\cdots\eand Fx_n \eand \bigl[Fy \eif (y=x_1 \eor \cdots \eor y=x_n)\bigr] \eand x_1 \neq x_2 \eand\cdots\eand x_{n-1}\neq x_n\Bigr)$ 
\end{ekey}

\subsection*{Especificando o tamanho do UD}

Remover $F$ das simbolizações acima produz sentenças que falam sobre o tamanho do UD. Por exemplo, `existem pelo menos 2 coisas (no UD)' pode ser simbolizado como $\exists x\exists y(x \neq y)$.

%  BEGIN: Rules of proof
% change margins so that all the rules will fit
\setlength{\topmargin}{0 in}
\setlength{\headheight}{0 in}
\setlength{\headsep}{0 in}
\setlength{\textheight}{9 in}
\setlength{\evensidemargin}{0.25 in}
\setlength{\oddsidemargin}{0.25 in}
\setlength{\textwidth}{6 in}
\newpage
% This starts a new page and skips a page if necessary so as
% to start on an even numbered page.
% That way, the rules of proof will be on facing pages.
% It fills it in with a somewhat gratuitous reference table.
\ifthenelse{\isodd{\thepage}}{
%	\ \vspace{2 in}\par\centerline{[ Esta página intencionalmente deixada em branco. ]}
\begin{table}
	Às vezes é mais fácil demonstrar algo fornecendo provas do que fornecendo modelos. Às vezes é o contrário.
	\begin{center}
	\begin{tabular*}{\textwidth}{p{10em}|p{10em}|p{10em}|}
	\cline{2-3}
	 & {\centerline{SIM}} & {\centerline{NÃO}}\\
	\cline{2-3}
	\script{A} é uma tautologia? & prove $\vdash\script{A}$ & dê um modelo no qual \script{A} é falsa\\
	\cline{2-3}
	\script{A} é uma contradição? &  prove $\vdash\enot\script{A}$ & dê um modelo no qual \script{A} é verdadeira\\
	\cline{2-3}
	\script{A} é contingente? & dê um modelo no qual \script{A} é verdadeira e outro no qual \script{A} é falsa & prove $\vdash\script{A}$ ou $\vdash\enot\script{A}$\\
	\cline{2-3}
	\script{A} e \script{B} são equivalentes? & prove \mbox{$\script{A}\vdash\script{B}$} e \mbox{$\script{B}\vdash\script{A}$}  & dê um modelo no qual \script{A} e \script{B} têm valores verdade diferentes\\
	\cline{2-3}
	O conjunto \model{A} é consistente? & dê um modelo no qual todas as sentenças em \model{A} são verdadeiras & tomando as sentenças em \model{A}, prove \script{B} e \enot\script{B}\\
	\cline{2-3}
	O argumento \mbox{`\script{P}, \therefore\ \script{C}'} é válido? & prove $\script{P}\vdash\script{C}$ & dê um modelo no qual \script{P} é verdadeira e \script{C} é falsa\\
	\cline{2-3}
	\end{tabular*}
	\end{center}
\end{table}
	
	\newpage
}{}
% eliminate page numbers
\pagestyle{empty}
\twocolumn

\label{ProofRules}
{\LARGE \bf Regras Básicas de Prova}

\textsc{Reiteração}

\begin{proof}
	\have[m]{a}{\script{A}}
	\have[\ ]{c}{\script{A}} \by{R}{a}
\end{proof}


\textsc{Introdução da Conjunção}

\begin{proof}
	\have[m]{a}{\script{A}}
	\have[n]{b}{\script{B}}
	\have[\ ]{c}{\script{A}\eand\script{B}} \ai{a, b}
\end{proof}

\textsc{Eliminação da Conjunção}

\begin{proof}
	\have[m]{ab}{\script{A}\eand\script{B}}
	\have[\ ]{a}{\script{A}} \ae{ab}
\end{proof}

\begin{proof}
	\have[m]{ab}{\script{A}\eand\script{B}}
	\have[\ ]{b}{\script{B}} \ae{ab}
\end{proof}

\textsc{Introdução da Disjunção}

\begin{proof}
	\have[m]{a}{\script{A}}
	\have[\ ]{ab}{\script{A}\eor\script{B}}\oi{a}
\end{proof}

\begin{proof}
	\have[m]{a}{\script{A}}
	\have[\ ]{ba}{\script{B}\eor\script{A}}\oi{a}
\end{proof}

\textsc{Eliminação da Disjunção}

\begin{proof}
	\have[m]{ab}{\script{A}\eor\script{B}}
	\have[n]{nb}{\enot\script{B}}
	\have[\ ]{a}{\script{A}} \oe{ab,nb}
\end{proof}

\begin{proof}
	\have[m]{ab}{\script{A}\eor\script{B}}
	\have[n]{na}{\enot\script{A}}
	\have[\ ]{b}{\script{B}} \oe{ab,na}
\end{proof}


\textsc{Introdução do Condicional}

\nopagebreak
\begin{proof}
	\open
		\hypo[m]{a}{\script{A}} \by{deseja \script{B}}{}
		\have[n]{b}{\script{B}}
	\close
	\have[\ ]{ab}{\script{A}\eif\script{B}}\ci{a-b}
\end{proof}

\pagebreak
\textsc{Eliminação do Condicional}

\begin{proof}
	\have[m]{ab}{\script{A}\eif\script{B}}
	\have[n]{a}{\script{A}}
	\have[\ ]{b}{\script{B}} \ce{ab,a}
\end{proof}

\textsc{Introdução do Bicondicional}

\begin{proof}
	\open
		\hypo[m]{a1}{\script{A}} \by{deseja \script{B}}{}
		\have[n]{b1}{\script{B}}
	\close
	\open
		\hypo[p]{b2}{\script{B}} \by{deseja \script{A}}{}
		\have[q]{a2}{\script{A}}
	\close
	\have[\ ]{ab}{\script{A}\eiff\script{B}}\bi{a1-b1,b2-a2}
\end{proof}

\textsc{Eliminação do Bicondicional}

\begin{proof}
	\have[m]{ab}{\script{A}\eiff\script{B}}
	\have[n]{a}{\script{B}}
	\have[\ ]{b}{\script{A}} \be{ab,a}
\end{proof}

\begin{proof}
	\have[m]{ab}{\script{A}\eiff\script{B}}
	\have[n]{a}{\script{A}}
	\have[\ ]{b}{\script{B}} \be{ab,a}
\end{proof}



\textsc{Introdução da Negação}

\begin{proof}
	\open
		\hypo[m]{a}{\script{A}} \by{para redução ao absurdo}{}
		\have[n][-1]{b}{\script{B}}
		\have{nb}{\enot\script{B}}
	\close
	\have[\ ]{na}{\enot\script{A}}\ni{a-nb}
\end{proof}

\textsc{Eliminação da Negação}

\begin{proof}
	\open
		\hypo[m]{na}{\enot\script{A}} \by{para redução ao absurdo}{}
		\have[n][-1]{b}{\script{B}}
		\have{nb}{\enot\script{B}}
	\close
	\have[\ ]{a}{\script{A}}\ne{na-nb}
\end{proof}








\newpage

{\LARGE \bf Regras dos Quantificadores}

\textsc{Introdução do Existencial}

\begin{proof}
	\have[m]{a}{\script{A}\script{c}}
	\have[\ ]{c}{\exists \script{x}\script{A}\script{x}} \Ei{a}
\end{proof}

Note que \script{x} pode substituir algumas ou todas as ocorrências de \script{c} em \script{A}\script{c}.



\textsc{Eliminação do Existencial}

\begin{proof}
	\have[m]{a}{\exists \script{x}\script{A}\script{x}}
	\open	
		\hypo[n]{b}{\script{A}\script{c}^\ast}
		\have[p]{c}{\script{B}}
	\close
	\have[\ ]{d}{\script{B}} \Ee{a,b-c}
\end{proof}

$^\ast$ \script{c} não deve aparecer em $\exists\script{x}\script{A}\script{x}$, em \script{B}, ou em qualquer suposição não descarregada.

\textsc{Introdução do Universal}

\begin{proof}
	\have[m]{a}{\script{A}\script{c}^\ast}
	\have[\ ]{c}{\forall \script{x}\script{A}\script{x}} \Ai{a}
\end{proof}

$^\ast$ \script{c} não deve ocorrer em qualquer suposição não descarregada.


\textsc{Eliminação do Universal}

\begin{proof}
	\have[m]{a}{\forall \script{x}\script{A}\script{x}}
	\have[\ ]{c}{\script{A}\script{c}} \Ae{a}
\end{proof}




{\LARGE \bf Regras de Identidade}

\begin{proof}
	\have[\ \,\,\,]{x}{\script{c}=\script{c}} \by{=I}{}
\end{proof}

\begin{proof}
	\have[m]{e}{\script{c}=\script{d}}
	\have[n]{a}{\script{A}}
	\have[\ ]{ea1}{\script{A}{c}\circlearrowleft{d}} \by{=E}{e,a}
\end{proof}

Uma constante pode substituir algumas ou todas as ocorrências da outra.





\newpage

{\LARGE \bf Regras Derivadas}

\textsc{Dilema}

\begin{proof}
	\have[m]{ab}{\script{A}\eor\script{B}}
	\have[n]{ac}{\script{A}\eif\script{C}}
	\have[p]{bc}{\script{B}\eif\script{C}}
	\have[\ ]{a}{\script{C}} \by{DIL}{ab,ac,bc}
\end{proof}

\textsc{Modus Tollens}

\begin{proof}
	\have[m]{ab}{\script{A}\eif\script{B}}
	\have[n]{a}{\enot\script{B}}
	\have[\ ]{b}{\enot\script{A}} \by{MT}{ab,a}
\end{proof}

\textsc{Silogismo Hipotético}

\begin{proof}
	\have[m]{ab}{\script{A}\eif\script{B}}
	\have[n]{bc}{\script{B}\eif\script{C}}
	\have[\ ]{ac}{\script{A}\eif\script{C}}\by{HS}{ab,bc}
\end{proof}



{\LARGE \bf Regras de Substituição}
{
\center

\textsc{Comutatividade} (Comm)\\
$(\script{A}\eand\script{B}) \Longleftrightarrow (\script{B}\eand\script{A})$\\
$(\script{A}\eor\script{B}) \Longleftrightarrow (\script{B}\eor\script{A})$\\
$(\script{A}\eiff\script{B}) \Longleftrightarrow (\script{B}\eiff\script{A})$

\textsc{DeMorgan} (DeM)\\
$\enot(\script{A}\eor\script{B}) \Longleftrightarrow (\enot\script{A}\eand\enot\script{B})$\\
$\enot(\script{A}\eand\script{B}) \Longleftrightarrow (\enot\script{A}\eor\enot\script{B})$

\textsc{Dupla Negação} (DN)\\
$\enot\enot\script{A} \Longleftrightarrow \script{A}$

\textsc{Condicional Material} (MC)\\
$(\script{A}\eif\script{B}) \Longleftrightarrow (\enot\script{A}\eor\script{B})$\\
$(\script{A}\eor\script{B}) \Longleftrightarrow (\enot\script{A}\eif\script{B})$

\textsc{Troca do Bicondicional} ({\eiff}{ex})\\
$[(\script{A}\eif\script{B})\eand(\script{B}\eif\script{A})] \Longleftrightarrow (\script{A}\eiff\script{B})$

\textsc{Negação de Quantificador} (QN)\\
$\enot\forall\script{x}\script{A} \Longleftrightarrow \exists\script{x}\enot\script{A}$\\
$\enot\exists\script{x}\script{A} \Longleftrightarrow \forall\script{x}\enot\script{A}$

}
