%!TEX root = forallx.tex
\thispagestyle{empty}
{\Huge\textcolor{burntorange}{Math}\forallx}

{\tt Introdução à Lógica Formal\\ para a \textcolor{burntorange}{Matemática}}

\vfill

{\sf \textbf{Edição Original:}}\\
{\sf P.D. Magnus}\\
\emph{University at Albany, State University of New York}

{\sf \textbf{Tradução e Adaptação:}}\\
{\sf Carlos André Duarte Costa}\\
\emph{Universidade Estadual de Alagoas}

\vfill

{\sf
	\href{https://www.fecundity.com/logic/}{fecundity.com/logic}, versão 1.4 [\bookversion]\\
	Versão em português: 13 de novembro de 2025
	Este livro é distribuído sob uma licença Creative Commons.\\
	(Atribuição 4.0 Internacional)
}

\newpage
\thispagestyle{empty}%
{\sf
O autor original, P.D. Magnus, gostaria de agradecer às pessoas que tornaram este projeto possível. Entre estas, destacam-se Cristyn Magnus, que leu muitas versões preliminares; Aaron Schiller, que foi um dos primeiros a utilizar o material e forneceu comentários consideráveis e úteis; e Bin Kang, Craig Erb, Nathan Carter, Wes McMichael, Selva Samuel, Dave Krueger, Brandon Lee, Toan Tran, Marcus Adams, Matthew Brown, e os alunos de Introdução à Lógica, que detectaram vários erros em versões anteriores do livro.

Carlos André Duarte Costa, que modificou a edição de Magnus para criar esta versão, gostaria, em primeiro lugar, de agradecer a P.D. Magnus pelo trabalho original e por disponibilizar este excelente material de forma aberta. Também estende seus agradecimentos aos colegas e estudantes que contribuíram com sugestões para esta adaptação em português.
}

\vfill
{
\copyright\ \ifthenelse{\year=2005}{\number\year}{2005--\number\year} P.D. Magnus (obra original)\\
\copyright\ \ifthenelse{\year=2024}{\number\year}{2024--\number\year} Carlos André Duarte Costa (tradução e adaptação)\\
Alguns direitos reservados.
}

{\footnotesize
\textbf{Você tem a liberdade de:} copiar, distribuir, exibir e executar a obra e criar obras derivadas.

\textbf{Sob as seguintes condições:} Atribuição. Você deve dar crédito ao autor original, da forma especificada pelo autor ou licenciante. Para qualquer reutilização ou distribuição, você deve deixar claro para outros os termos de licença desta obra. Qualquer uma dessas condições pode ser renunciada, desde que você obtenha permissão do autor. Os seus direitos de uso justo e outras utilizações não são afetados pelo acima exposto.

\textbf{Este é um sumário legível por humanos do contrato de licença completo disponível em:} \url{https://creativecommons.org/licenses/by/4.0/deed.pt}
}

{
A diagramação foi realizada inteiramente em \LaTeX$2\varepsilon$. O estilo para diagramação de provas é baseado em fitch.sty (v0.4) de Peter Selinger, University of Ottawa.

Esta cópia de \forallx\ (versão em português) foi \textbf{atualizada em 13 de novembro de 2025}. A versão mais recente da obra original está disponível em \url{http://www.fecundity.com/logic}
}
